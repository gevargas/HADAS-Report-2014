% Pour inclure une illustration utiliser la macro suivante
% \dopdffig{ficher}{largeur}{legende}{label pour reference}
% exemple : \dopdffig{graph1.pdf}{.6\textwidth}{Resultats de l'annee}{fig:res_annee}

% {\'e} {\`e}
% {\`a}

\section{Equipe HADAS - Axe Traitement de Donn{\'e}es et de Connaissances {\`a} Grande Echelle} 
\index{HADAS}
\label{sec:hadas}

%==========================================================================
\subsection{Scientific Presentation} % (fold)
\label{sub:hadas_scientific_presentation}
%==========================================================================

% 1 page pour la description
%  Research group members
%  Group evolution in terms of members
% . .. .. .. .. .. .. .. .. .. .. .. .. .. .. .. .. .. .. .. .. .. .. .. .. .. .. .. .. .. .. .. .. .. .. .. .. .. .. .. .. .. .. .. .. .. .. .. .. .. .. .. .. . 
\paragraph{Research group:}
% . .. .. .. .. .. .. .. .. .. .. .. .. .. .. .. .. .. .. .. .. .. .. .. .. .. .. .. .. .. .. .. .. .. .. .. .. .. .. .. .. .. .. .. .. .. .. .. .. .. .. .. .. . 

The HADAS (Heterogeneous  Autonomous Distributed data Services) group was founded in october 2005 as a new team for the LIG laboratory. It actually follows the STORM team, directed by M. Adiba until 1996. 
Over the years, the group  proposed an evolution of the scientific vision of a database management systems  as  a semantic-based infrastructure for managing ubiquitous and heterogeneous data services. 
The team currently includes 9  permanent research people (2 full professors, 5 Associate Professor and two CNRS Research Scientist),  a research engineer and 15 PHD students and post-docs. During this period we hired 4 people. The following table lists the permanent people during the period. 

% \textit{(Reference date is Oct. 1st 2009-june 2014) 1 page}\\
% \textit{(including outgoing doctoral students who have not yet presented their thesis, even if listed as departing in OSE). }

%\begin{table}[ht]

\begin{center}\begin{tabular}{|p{2.5cm}|*{4}{c|}}
%\begin{center}\begin{tabular}{|p{3cm}|m{3cm}|m{3cm}|m{2cm}|m{2cm}|}
%\caption{List of permanent faculty (Oct 1st 2009 - june 2014)} % title of Table
%\centering % used for centering table
%\begin{tabular}{c c c c c }
\hline 
Name & First name & Function & Institution & Arrival date\\\hline \hline % inserts table
Adiba & Michel & External collaborator  & UJF & Oct 2010\\\hline
Amer Yahia & Sihem & Research Scientist & CNRS & Dec 2011  \\\hline
Bobineau & Christophe & Associate Professor & Grenoble INP & Sep 2003 \\\hline
Collet & Christine & Full Professor & Grenoble INP & Sep 1999 \\\hline
Dubl{\'e} & Etienne & ITA  & CNRS & May 2011\\\hline
Ibrahim & Noha & Associate Professor & Grenoble INP & Sep 2010  \\\hline
Jouanot & Fabrice & Associate Professor & UJF & Sep 2003 \\\hline
Leroy & Vincent & Associate Professor & UJF & Sep 2012 \\\hline
Rousset & Marie-christine & Full Professor & UJF & Sep 2005 \\\hline
Termier & Alexandre & Associate Professor & UJF & Sep 2007 \\\hline
Vargas-solar & Genoveva & Research Scientist & CNRS & Jan 2002 \\\hline
%\definecolor{Gray}{gray}{0.85}
%Name & \cellcolor{grisclair}First name & \cellcolor{grisclair}Function & \cellcolor{grisclair}Institution & \cellcolor{grisclair}Arrival date \\\hline
\hline
\end{tabular}\end{center}
\makeatother
%\end{table}

% \textit{(Free form description of how the team composition has evolved: up to 15 lines)}\\

\paragraph{Group evolution:}
Research activities on services has been reinforced in the group with the arrival of Noha Ibrahim, associate professor who joined the team in September 2010. 
Then, the arrival of Sihem Amer-Yahia Research Director (Dec 2011) and Vincent Leroy Associate Professor UJF / Research scientist (Sept 2012) brought new areas of research in the group centered around large scale data processing on the social Web.
Etienne Dubl{\'e} Research Engineer (Mai 2011) helps us in developing and finalizing some of our prototypes. He also brought expertises in new kind of sensor networks and of file systems. 

% Research description  themes 
% subsection scientific_presentation (end)
% . .. .. .. .. .. .. .. .. .. .. .. .. .. .. .. .. .. .. .. .. .. .. .. .. .. .. .. .. .. .. .. .. .. .. .. .. .. .. .. .. .. .. .. .. .. .. .. .. .. .. .. .. . 
\paragraph{Research description:}
%\
% . .. .. .. .. .. .. .. .. .. .. .. .. .. .. .. .. .. .. .. .. .. .. .. .. .. .. .. .. .. .. .. .. .. .. .. .. .. .. .. .. .. .. .. .. .. .. .. .. .. .. .. .. . 

The HADAS group has contributed in the following areas: relational data models, snapshots and their semantics, active and temporal databases and object-oriented database systems. The advent of the web and middleware infrastructures in the early 1990s has profoundly changed the nature of research in databases. 
Our research is related to the changes in devices and softwares:
\begin{itemize}
\item Memories and disks with more capacity and faster access
\item Faster processors and networks
\item Algorithmic advances, e.g. parallel computing
\item Cloud computing (virtualization, elasticity, pay-as-you-go ...)
\end{itemize}

More recently big data management and analysis introduces  more challenging perspectives. Technological changes have reduced the cost of creating, capturing, managing and storing information to a sixth of what it was in 2005. This allowed a scale change in the size of data, distribution of data, number of connected devices, and number of users. 

% Considering this globalization of data, knowledge and computing resources and given the quality and intelligence of the expected data management functions, we have been confronted (and are still confronted) with a change in the field of databases.

To face these challenges we decided to revisit database systems and consider them not anymore as centralized data storage systems but as data management services largely distributed and deployed over different types of large scale systems (grids, peer-to-peer networks, sensor networks, ambient and ubiquitous environments). 
Semantics is at the heart of this approach as it is used at all levels of the process of designing or composing data services for handling autonomy, dynamic behavior and heterogeneity of both users and data sources. 

The activities of the group during these past years have been centered on the following themes:
\begin{itemize}
\item  Accessing data in large-scale systems: a first aspect concerns query optimization in distributed and dynamic systems ; a second aspect deals with mining large amounts of data to extract patterns of interest.
\item  Composing data services in a dynamic way:  we investigate models, algorithms and tools for coordinating services with non functional properties (contracts) and for providing access to heterogeneous data coming from services
\item  Reasoning on data semantics: we investigate different models and  algorithms  for querying data (or resources) through  possibly heterogeneous and distributed ontologies.
\end{itemize}

We participated to the Optimacs and Ubiquest ANR projects that bridge the gap between data management and (web) services and, also  between networks and data management.   
We have been involved in the Continuum, Dataring and Qualinca projects that exploit the semantics for relating data, devices and services in ubiquitous environments and peer-to-peer data management systems. In the PAGODA project, we collaborate with the LaDAF (Laboratoire d'Anatomie de Grenoble) to enrich anatomical 3D graphical models with ontologies using Semantic Web and Linked Data technologies. We collaborate with industry in several projects. The Datalyse projets involves several partners interested in Big Data (Business \& Decision, Eolas, supermarkets), its objective is to build a smart warehouse demonstrator for the collection, analysis, integration and enrichment of heterogeneous Big Data  of type "Big Data User" (UBD) or from machines, of type "Big Data Monitoring" (MBD).   We also have established strong ties with STMicroelectronics, especially in the context of the SoCTrace projet for analyzing execution traces with data mining methods. In the framework of the ADEME SoGrid project we collaborate (with more than 10 companies) on  event flow management systems for Smartgrids.  In addition,  we recently started ALICIA, an ANR project on crowdsourcing and are strongly involved in the creation of a GDR on Big Data and Data Sciences.  

Results of our research have direct impact on applications dealing with huge amounts of data and resources largely distributed in pervasive environments, such as data spaces, smart grids and smart buildings, hardware and software observation and social networks.

%==========================================================================
\subsection{Scientific and Technological Results} % (fold)
\label{sub:hadas_scientific_and_technological_results}
%==========================================================================
% Critere C1 : Qualite scientifique et production
% 2 pages
%  resultats scientifiques majeurs

%. - . -. - . -. - . -. - . -. - . -. - . -. - . -. - . -. - . -. - . -. - . -. - . -. - . -. - . -. - . -. - . -. - . -. - . -. - . -. - . -. - . -
\subsubsection{Accessing data in large-scale systems}
\label{optimisation}
%. - . -. - . -. - . -. - . -. - . -. - . -. - . -. - . -. - . -. - . -. - . -. - . -. - . -. - . -. - . -. - . -. - . -. - . -. - . -. - . -. - . -

Accessing data concerns several aspects of large scale systems: number of resources, data volume and data complexity. It basically means using declarative queries that are optimized based on system characteristics.
Data mining is another way to query large quantities of data, by extracting interesting patterns from them. Such patterns provide meaningful abstractions of raw data, which are thus more appropriate for data analysis. 
Globally, the difficulty for evaluating queries efficiently on nowadays applications motivates this work to revisit traditional query optimization techniques. The following presents these two aspects of accessing data in the large. It also focuses on works done on querying the social web. 

%. - . -. - . -. - . -. - . -. - . -. - . -. - . -. - . -. - . -. - . -. - . -. - . -. - . -. - . -. - . -. - . -. - . -. - . -. - . -. - . -. - . -
\subsubsection*{1- CBR query optimization}
%. - . -. - . -. - . -. - . -. - . -. - . -. - . -. - . -. - . -. - . -. - . -. - . -. - . -. - . -. - . -. - . -. - . -. - . -. - . -. - . -. - . -

Our research contributes to the development of new distributed query optimization techniques. It relies on the adaptation of machine learning, more precisely Case-Based Reasoning(CBR), and pseudo random search space exploration (also exploiting the case base) to produce efficient  query execution plans according to application specific optimization objectives expressed over resource consumption (e.g. time, energy, number of messages).  The query plan generation considered multiple optimization objectives customizable to application requirements (QoS based Hybrid Query optimization).
This research led to the following original contributions:
\begin{itemize}
\item  A query optimization approach that use cases generated from the evaluation of similar past queries. A query case comprises: (i) the query (the problem), (ii) the query plan (the solution) and (iii) the measures of computational resources consumed during the query plan execution (the evaluation of the solution). 

\item  A query plan generation process [1] that uses classical query optimization heuristics and makes decisions randomly when information on data is not available (e.g. for ordering joins, selecting algorithms or choosing message exchange protocols). This process also exploits the CBR principle for generating plans for subqueries, thus accelerating the learning of new cases. 

\item  A Simulation Platform [2] allowing to experiment distributed query optimization and rule-based programs over a set of distributed data-enabled devices hosting virtual machines(VM). A VM integrates a query optimization engine [3] implementing the above techniques. 
\end{itemize}

\begin{description}
\item[Main contracts:] \ \\
- UBIQUEST(2009-2013) ANR- Programme BLANC. Ubiquitous Quest: a declarative approach for integrated network and data management in wireless multi-hop networks. \\
- OPTIMACS (2009-2012) ANR-Programme ARPEGE. Service composition based framework for optimizing queries.\\
- CAISES (2012 - 2015) European Union FP7, IRESES program. Observation and industrial management on the cloud.  \\
- SOGrid (2013-2017) ADEME Le r{\'e}seau {\'e}lectrique de demain. 

\item[Key references:]~% (if necessary)
%\cite{di:bbg+09} 

[1]	Lourdes Martinez, Christine Collet, Christophe Bobineau, Etienne Dubl{\'e}. The QOL approach for optimizing distributed queries without complete knowledge. IDEAS, 91-99, 2012

[2] Ahmad Ahmad-Kassem, Christophe Bobineau, Christine Collet, Etienne Dubl{\'e}, St{\'e}phane Grumbach, FudaMa, Lourdes Martinez, St{\'e}phane Ub{\'e}da. UBIQUEST, for rapid prototyping of networking applications. IDEAS, 187-192, 2012

[3] Lourdes Martinez, Christine Collet, Christophe Bobineau and Etienne Dubl{\'e}. CoBRa for optimizing global queries. BDA, 2013

%[4] Carlos-Manuel Lopez-Enriquez, Genoveva Vargas-Solar, Jos{\'e}-Luis Zechinelli-Martini, Christine Collet. Hybrid query generation,  LANMR,  117-128, 2012

\end{description}

%. - . -. - . -. - . -. - . -. - . -. - . -. - . -. - . -. - . -. - . -. - . -. - . -. - . -. - . -. - . -. - . -. - . -. - . -. - . -. - . -. - . -
\subsubsection*{2- Data mining}
%. - . -. - . -. - . -. - . -. - . -. - . -. - . -. - . -. - . -. - . -. - . -. - . -. - . -. - . -. - . -. - . -. - . -. - . -. - . -. - . -. - . -

Data mining is the automatic extraction of unknown and potentially interesting information from large quantities of data. One of the major fields of data mining consists in discovering patterns occurring frequently (i.e. more than a given threshold) in data. 
%The data can either be unstructured data (sets of items like supermarket transactions for example) or data having a sequence, tree or graph structure (graph structured molecules for example).

This research focused on improving frequent pattern mining algorithms, both to make them more scalable and to apply them to real data analysis contexts. Main results are:

\begin{itemize}
\item  From the scalability point of view, we acquired a strong expertise on exploiting multicore processors for pattern mining [2, 5].
The proposed algorithms are also based on the notion of closed patterns [2, 4, 5], reducing the output size (hence the computation time) without loss of information.

\item These works culminated with the proposition of ParaMiner [2], the first parallel and generic algorithm for mining closed patterns.

\item From an application point of view, works have been done on the analysis of execution traces, in collaboration with STMicroelectronics.
They have improved the way to discover periodic behaviors and their disruption in traces [4], to rewrite a trace with a few significant sequences of events [3], and to automatically discover hotspots of memory contention in a parallel code [1].
\end{itemize}

\begin{description}

\item[Main contracts:] \ \\
- FUI SoCTrace that  aims to develop a set of methods and tools based on traces of execution produced by multi-core embedded applications. 
  
\item[Key references:]~% (if necessary)
%\cite{di:bbg+09} 

[1] Sofiane Lagraa, Alexandre Termier, Fr{\'e}d{\'e}ric P{\'e}trot: Data mining MPSoC simulation traces to identify concurrent memory access patterns. DATE 2013: 755-760

[2] Benjamin N{\'e}grevergne, Alexandre Termier, Marie-Christine Rousset, Jean-Francois M{\'e}haut: ParaMiner: a generic pattern mining algorithm for multi-core architectures, Data Mining and Knowledge Discovery, 2013

[3] Christiane Kamdem Kengne, Leon Constantin Fopa, Alexandre Termier, Noha Ibrahim, Marie-Christine Rousset, Takashi Washio, Miguel Santana: Efficiently rewriting large multimedia application execution traces with few event sequences. KDD 2013: pp 1348-1356

[4] Patricia Lopez-Cueva, Aur{\'e}lie Bertaux, Alexandre Termier, Jean-Francois M{\'e}haut, Miguel Santana: Debugging Embedded Multimedia Application Traces through Periodic Pattern Mining, EMSOFT 2012 : pp 13-22

[5] Trong Dinh Thac Do, Anne Laurent, Alexandre Termier: PGLCM: Efficient Parallel Mining of Closed Frequent Gradual Itemsets, ICDM 2010 : pp 138-147

[6] Sofiane Lagraa, Alexandre Termier, Frédéric Pétrot: Scalability Bottlenecks Discovery in MPSoC Platforms Using Data Mining on Simulation Traces, Design Automation and Test in Europe Conference (DATE), 2014, to appear.

\end{description}

%. - . -. - . -. - . -. - . -. - . -. - . -. - . -. - . -. - . -. - . -. - . -. - . -. - . -. - . -. - . -. - . -. - . -. - . -. - . -. - . -. - . -
\subsubsection*{3- Social Web Data access}
%. - . -. - . -. - . -. - . -. - . -. - . -. - . -. - . -. - . -. - . -. - . -. - . -. - . -. - . -. - . -. - . -. - . -. - . -. - . -. - . -. - . -

Research has been done on new exploration problems to find useful user groups in collaborative rating datasets [1,2,4] and useful information in online news [3,5]. Our formulation of exploration as an optimization problem where various dimensions such as similarity, diversity, and coverage are optimized, leads to reductions from well-known problems and adaptations of well-established algorithms. Large-scale user studies have been conducted to verify the effectiveness of our findings. The current research direction is to blend efficient mining with exploration and to develop an evaluation methodology for large-scale information exploration.

Social Web Data access also concerns optimization. Data is stored within data centers, which constitute distributed systems. It is therefore important to optimize communications between the machines to avoid saturating the network equipment. A first work considered the problem of data routing between users of social networks. The key idea was to identify hubs that aggregate data from several sources and reduce the number of messages exchanged [6]. A second work considers the problem of data placement in hierarchical network structures. A reactive algorithm monitors data access patterns to identify locations in which new replicas of data should be deployed to reduce routers saturation [7].

\begin{description}

\item[Main contracts:]  \ \\
- Datalyse(2013-2016): Big Data Models and Algorithms, Investissement d'Avenir.

- ALICIA (2014-2017) , Agence Nationale de la Recherche, France, Programme BLANC. 

- AGIR Big join (2013-2016): Programme AGIR Grenoble INP and  University Joseph Fourrier,  Mod{\`e}les et algorithmes pour les jointures de Big Data sur Map-Reduce. 
  
\item[Key references:]
~% (if necessary)
%\cite{di:bbg+09} 

[1]   Behrooz Omidvar Tehrani, Sihem Amer-Yahia, Alexandre Termier, Aur{\'e}lie Bertaux, Eric Gaussier, Marie-Christine Rousset: Towards a Framework for Semantic Exploration of Frequent Patterns. IMMoA 2013: 7-14 (workshops) 

[2]  Mikalai Tsytsarau, Sihem Amer-Yahia, Themis Palpanas: Efficient sentiment correlation for large-scale demographics. SIGMOD Conference 2013: 253-264

[3]  Sofiane Abbar, Sihem Amer-Yahia, Piotr Indyk, Sepideh Mahabadi: Real-time recommendation of diverse related articles. WWW 2013: 1-12

[4]  Mahashweta Das, Saravanan Thirumuruganathan, Sihem Amer-Yahia, Gautam Das, Cong Yu: Who Tags What? An Analysis Framework. PVLDB (11): 1567-1578 (2012)

[5]  Demo: Sihem Amer-Yahia, Samreen Anjum, Amira Ghenai, Aysha Siddique, Sofiane Abbar, Sam Madden, Adam Marcus, Mohammed El-Haddad: MAQSA: a system for social analytics on news. SIGMOD Conference 2012: 653-656

[6] Aristides Gionis, Flavio P. Junqueira, Vincent Leroy, Marco Serafini and Ingmar Weber: Piggybacking on social networks. In Proceedings of the 39th International Conference on Very Large Databases (VLDB), pages 409-420, 2013

[7]  Xiao Bai, Arnaud J{\'e}gou, Flavio P. Junqueira and Vincent Leroy:  DynaSoRe: Efficient In-Memory Store for Social Applications. In Proceedings of the 14th International Middleware Conference (Middleware)  pages 425-444, 2013

\end{description}

%. - . -. - . -. - . -. - . -. - . -. - . -. - . -. - . -. - . -. - . -. - . -. - . -. - . -. - . -. - . -. - . -. - . -. - . -. - . -. - . -. - . -
\subsubsection{Composing data services on the fly }
%. - . -. - . -. - . -. - . -. - . -. - . -. - . -. - . -. - . -. - . -. - . -. - . -. - . -. - . -. - . -. - . -. - . -. - . -. - . -. - . -. - . -

Composing services  exported by different organisations is a key issue when building  large scale and data-intensive systems. 
Composition must take into account the characteristics of  execution environments (e.g., memory and computing, and network capabilities) to dynamically  compose  services, and then to adapt  compositions depending on the availability and evolution of services. 
 Compositions can be executed  on platforms providing  unlimited resources through a "pay as U go model", aware of energy consumption or services reputation, provenance, availability, and reliability[2].  These features  guide the way compositions are specified and executed for fulfilling given user requirements and preferences[3]. These properties are modeled as  non functional aspects and QoS (quality of service) criteria  that can provide guarantees to the execution of compositions and to the way results are delivered.

Our research in this topic contributes to the construction of service based data management systems as service compositions. Once data management is delivered as a service, it can have associated non-functional properties.
We proposed methodologies, algorithms, languages and tools for designing and executing  service compositions with non-functional properties expressed as policies. 

We  applied our approach for the efficient evaluation  of queries as coordinations of services, including data  and computing services.  
This lead to the following results: 
\begin{itemize}
\item  an Hybrid query model for expressing queries as data service coordinations based on workflows. The approach uses the abstract state machines (ASM) formalism for defining the model[1].

\item  a query language HSQL (Hybrid Services Query Languages) associated to the hybrid query model and the language MQLiST (Mashup Query Spatio Temporal Language) for integrating hybrid query results in a mashup. Both are extensions of SQL[5]. 

\item  an algorithm BP GYO for generating the query workflow that implements a query expressed in HSQL[4].

% An algorithm for computing computing the query workflows search space that implement an hybrid query and that respect an associated SLA expressed as an aggregation of measures (economic, temporal and energetic costs).

\item an hybrid query evaluation engine HYPATIA[1].

\item an Active Policy model and language for specifying the QoS properties to be associated to service compositions modelled as workflows; and inforcement actions when they are not verified[5,6].

\end{itemize}


\begin{description}

\item[Main contracts:] \ \\
- OPTIMACS (2009-2012)	ANR-Programme ARPEGE. Service composition based framework for optimizing queries. \\
- CLEVER (2011-2013) STICAMSUD program  U. de la Rep{\'u}blica, Uruguay, UFRN Brazil, France. Environment virtual observatory on cloud. \\
- SWANS (2014-2016) CNRS STiC-AMSUD Program U. de la Rep{\'u}blica, Uruguay, UFRN Brazil,  France. \\
- Keystone - COST ICT Program (2013-2014) : Semantic keyword-based search on structured data sources. \\
- AIWS (2012 - 2014), PEPS  CNRS program. Discovering conversations among services by analyzing event logs.\\
- CAISES (2012 - 2015), European Union FP7, IRESES program (UK, France, Ukrania, China). Observation and industrial management on the cloud.  

\item[Key references:]~% (if necessary)
%\cite{di:bbg+09} 

[1] V. Cuevas-Vicenttin, G.  Vargas-Solar, C. Collet, Evaluating Hybrid Queries through Service Coordination in HYPATIA, In Proceedings of the 15th International Conference on Extending Database Technology (EDBT), Berlin, Germany, 2012

[2] T. Delot, S. Ilarri, M. Thilliez, G.  Vargas-Solar, S. Lecomte, Multi-scale query processing in vehicular networks, In Journal of Ambient Intelligence and Humanized Computing, Springer Verlag, ISSN 1868?5137, 2(3), 2011, pp. 213?226

[ 3]	Genoveva Vargas-Solar, Catarina Ferreira da Silva, Parisa Ghodous, Jos{\'e}-Luis Zechinelli-Martini, Moving energy consumption control into the cloud by coordinating services, International Journal of Computing Applications, Special Issue. December 2013.

[4] Carlos-Manuel Lopez-Enriquez, Genoveva Vargas-Solar, Jos{\'e}-Luis Zechinelli-Martini, Christine Collet, Hybrid query generation,  LANMR,  117128,2012.

[5] Valeria de Castro, Martin A. Musicante, Umberto Souza da Costa, Pl{\'a}cido A. de Souza Neto, and Genoveva Vargas-Solar, Supporting Non-Functional Requirements in Services Software Development Process: An MDD Approach, In Proceedings of the 40th International Conference on Current Trends in Theory and Practice of Computer Science,  LNCS Springer Verlag, High Tatras, Slovakia, January, 2014. 

[6] Javier A. Espinosa-Oviedo, Genoveva Vargas-Solar, Jos{\'e}-Luis Zechinelli-Martini, Christine Collet. Policy driven services coordination for building social networks based applications. In Proc. of the 8th Int. Conference on Services Computing (SCC'11), Work-in-Progress Track, Washington, DC, USA, July 2011.
\end{description}

%. - . -. - . -. - . -. - . -. - . -. - . -. - . -. - . -. - . -. - . -. - . -. - . -. - . -. - . -. - . -. - . -. - . -. - . -. - . -. - . -. - . -
\subsubsection{Reasoning on data semantics}
%. - . -. - . -. - . -. - . -. - . -. - . -. - . -. - . -. - . -. - . -. - . -. - . -. - . -. - . -. - . -. - . -. - . -. - . -. - . -. - . -. - . -

This research is focused on combining reasoning and data management for efficiently querying and linking Web data through ontologies. Ontologies are very useful in many applications to express domain-specific knowledge over data that may be incomplete, uncertain or even inconsistent because coming from autonomous data sources distributed over the Web. 

The proposed approach relies on recent complexity  results showing that the expressive power of ontologies must be limited for making tractable reasoning on data enriched with ontologies. In particular,  (several fragments of) the DL-Lite description logic  we have been studied  in the decentralized setting of P2P semantic networks. [1] designs a novel setting for robust module-based data management allowing to re-use a part of a  reference ontology-based data system as an independent module while guaranteeing that it evolves safely w.r.t both the reference  schema and its associated data. We are investigating how it applies to extract modules from the knowledge base on anatomy of My Corporis Fabrica.  [2,3]  proposed a novel model of trust based on alignments between taxonomies for guiding the query answering process in P2P semantic networks. Finally,  [4]  provides a novel method that us systematic and mathematically well-founded for discovering mappings between taxonomies of classes. 

% We are also collaborating with Alexandre Termier and Noha Ibrahim on pattern-mining and semantic trace analysis. 

\begin{description}
\item[Main contracts:] \ \\
- CONTINUUM (2008-2011), ANR - Programme R{\'e}seaux du futur et services.  It addresses the problem of service continuity within the long-term vision of ambient intelligence. \\
- DATARING (2008-2011), ANR - Programme R{\'e}seaux du futur et services on the problem of P2P data sharing for online communities. \\
- PAGODA (2013-2016)  ANR 12 JS02 007 01, Programme JCJC. The aim of this project is to develop practical algorithms for ontology-based data access. \\
- QUALINCA (2012-2016): ANR-2012-CORD-012. The aim of this project is to develop methods and algorithms for improving the quality and interoperability of large documentary catalogs. 
  
\item[Key references:]~% (if necessary)
%\cite{di:bbg+09} 

[1] Robust Module-based Data Management. Francois Goasdou and Marie-Christine Rousset. IEEE Transactions on Knowledge and Data Engineering , Volume 25, Issue 3, March 2013, pages 648-661. 

[2] Alignment-based trust for resource finding in semantic P2P networks. Manuel Atencia, Jerome Euzenat, Giuseppe Pirro and Marie-Christine Rousset. Proceedings of ISWC 2011 (10th International Semantic Web Conference). 

[3] Trust in Networks of Ontologies and Alignments. Manuel Atencia, Mustafa Al Bakri and Marie-Christine Rousset.  Knowledge and Information Systems, Volume 37, number 3, December 2013, 28 pages. 

[4] Discovery of Probabilistic Mappings between Taxonomies: Principles and Experiments Remi Tournaire, Jean-Marc Petit, Marie-Christine Rousset, and Alexandre Termier. Journal of Data Semantics (JoDS), Volume 15, pages 66-101. 

[5] Web Data Management, Serge Abiteboul, Ioana Manolescu, Philippe Rigaux, Marie-Christine Rousset, Pierre Senellart, book published by Cambridge University Press. 

\end{description}

%. - . -. - . -. - . -. - . -. - . -. - . -. - . -. - . -. - . -. - . -. - . -. - . -. - . -. - . -. - . -. - . -. - . -. - . -. - . -. - . -. - . -
\subsubsection{Publications} % (fold)
%. - . -. - . -. - . -. - . -. - . -. - . -. - . -. - . -. - . -. - . -. - . -. - . -. - . -. - . -. - . -. - . -. - . -. - . -. - . -. - . -. - . 

% tableau recap issu de pistou
%% 


{\bf This table has to be UPDATED }


The following table synthesizes the scientific publications of the group.
\begin{center}\begin{tabular}{|p{6cm}|*{7}{c|}}
\hline
~ &2009 &2010 &2011 &2012 &2013 &2014 & \textbf{Total} \\
\hline
International peer reviewed journal [ACL] & 3  & 3  & 0  & 4  & 3  & 1  & 14 \\
\hline
International peer-reviewed conference proceedings [ACT] & 9  & 16  & 4  & 15  & 15  & 1  & 60 \\
\hline
Short communications [COM] and posters [AFF] in conferences and workshops & 2  & 0  & 0  & 0  & 2  & 0  & 4 \\
\hline
Scientific books and chapter [OS] & 1  & 2  & 1  & 0  & 0  & 0  & 4 \\
\hline
National peer-reviewed conference proceedings [ACTN] & 2  & 0  & 0  & 0  & 0  & 0  & 2 \\
\hline
Book or Proceedings editing [DO] & 1  & 0  & 0  & 0  & 0  & 0  & 1 \\
\hline
Invited conferences [INV] & 5  & 2  & 1  & 5  & 5  & 1  & 19 \\
\hline
Doctoral Dissertations and Habilitations Theses [TH] & 2  & 2  & 2  & 0  & 3  & 3  & 12 \\
\hline
Other Publications [AP] & 0  & 1  & 4  & 1  & 2  & 1  & 9 \\
\hline \textbf{Total} &23&24&11&20&25&5 &108\\
\hline
\end{tabular}\end{center}
\makeatother

%% discussion de l'{\'e}volution des publications en terme de quantit{\'e} et qualit{\'e} entre le precedent quadriennal et 2009-2014. 

The number of doctoral students is rather stable over the last five years. We have 10PhD per year, all supervised by faculty members, resulting in 2 PhD's defense per year. 
%We have 10 PhD per year, all supervised by faculty members, resulting in 2 PhD�s defense per year. 

The numbers of publications in international conferences is significative and is stable for this past five years (14 per year) but the number of international publications per publishing scientific globally decreased. 
% increased ( past period: from to 2,14 to 2,3 per year; current period: ? ). 

% subsection scientific_and_technological_results (end)


%==========================================================================
\subsection{Visibility and attractivity} % (fold)
\label{sub:hadas_visibility_and_attractivity}
%==========================================================================

% Critere C2 : Rayonnement et attractivite  academiques
% 1 page
% - Rayonnement : honneur, nominations,

%. - . -. - . -. - . -. - . -. - . -. - . -. - . -. - . -. - . -. - . -. - . -. - . -. - . -. - . -. - . -. - . -. - . -. - . -. - . -. - . -. - . -
\subsubsection{Rayonnement}
%. - . -. - . -. - . -. - . -. - . -. - . -. - . -. - . -. - . -. - . -. - . -. - . -. - . -. - . -. - . -. - . -. - . -. - . -. - . -. - . -. - . -

\subsubsection*{Honor}
-  {\it Chevalier de l'ordre national du m{\'e}rite: }: M.-C. Rousset (2011), Ch. Collet (2011)

\subsubsection*{Nomination}
\begin{itemize}

\setlength{\itemindent}{-0.5cm}
\setlength{\itemsep}{-0.1cm}

\item {\it Membre  Institut Universitaire de France (IUF) }: M.-C. Rousset (2011-2016)

\item {\it Membre  du Comit\'e National du CNRS (nomm\'ee 2010-1012 dans l'ancienne section 7, \'elue dans la nouvelle section 6)}: M.-C. Rousset.

\item {\it Conseil scientifique de la chaire d'excellence Smart Grids entre Grenoble INP et ERDF (2012- )}: Ch. Collet.

\item {\it Comit{\'e} de pilotage ANR, Mod les num{\'e}riques (2010-2013)}: Ch. Collet.

\item {\it VP adjointe recherche groupe Grenoble INP}: Ch. Collet (April 2007-2012).

\item {\it Conseil scientifique INS2i - CNRS}: Ch. Collet (2010-).

\item {\it D\'el\'egu\'ee scientifique du LIG}: M.-C. Rousset.

\item {\it Jury du prix de th{\`e}se Gilles Kahn (2010-2012) (prix d{\'e}cern{\'e} par Specif et patronn{\'e} par l'Acad{\'e}mie des Sciences).}: Ch. Collet.

\item {\it Membre de la commission de sp{\'e}cialistes R{\'e}seau de technologies d'information CONACYT, Mexique}: G. Vargas-Solar.

\item {\it Charg\'ee de mission ``International'' aupr\`es du Coll\`ege Doctoral de l'Universit\'e de Grenoble}: M.-C. Rousset.

\item {\it Responsable scientifique et technique du Labex PERSYVAL-lab}: M.-C. Rousset.
 
% o   Animation de la strat�gie de coop�ration internationale du Mexique en TIC et prise de d�cisions des strat�gies scientifiques nationales en Informatique

% o   Taille du r�seau TIC 650 participants


\end{itemize}


%\subsection*{Prizes and Awards}

%. - . -. - . -. - . -. - . -. - . -. - . -. - . -. - . -. - . -. - . -. - . -. - . -. - . -. - . -. - . -. - . -. - . -. - . -. - . -. - . -. - . -
\subsubsection*{Best Paper Awards}
%. - . -. - . -. - . -. - . -. - . -. - . -. - . -. - . -. - . -. - . -. - . -. - . -. - . -. - . -. - . -. - . -. - . -. - . -. - . -. - . -. - . -
\begin{itemize}

\item Three prices at SSSW 12 (Summer School on Ontology Engineering and the Semantic Web): Mustafa Al Bakri  

\item Best Paper Award track Embedded Software, conference DATE 2014, Sofiane Lagraa ( HADAS/LIG and SLS/TIMA) 

\end{itemize}



% - Implications dans des comite s, expertises, recrutements ...]
% - Editorial board, program commitees evaluation commiRees, ...
% subsection visibility_and_attractivity (end)

%. - . -. - . -. - . -. - . -. - . -. - . -. - . -. - . -. - . -. - . -. - . -. - . -. - . -. - . -. - . -. - . -. - . -. - . -. - . -. - . -. - . -
\subsubsection{Contribution to the Scientific Community}
%. - . -. - . -. - . -. - . -. - . -. - . -. - . -. - . -. - . -. - . -. - . -. - . -. - . -. - . -. - . -. - . -. - . -. - . -. - . -. - . -. - . 

%. - . -. - . -. - . -. - . -. - . -. - . -. - . -. - . -. - . -. - . -. - . -. - . -. - . -. - . -. - . -. - . -. - . -. - . -. - . -. - . -. - . -
\subsubsection*{Management of Scientific Organisations}
%. - . -. - . -. - . -. - . -. - . -. - . -. - . -. - . -. - . -. - . -. - . -. - . -. - . -. - . -. - . -. - . -. - . -. - . -. - . -. - . -. - . -

\begin{itemize}
\setlength{\itemindent}{-0.5cm}
\setlength{\itemsep}{-0.1cm}

\item {\it President of the Extended Database Technology (EDBT) association}: Ch. Collet (2013 - )

\item {\it Member of the Extended Database Technology(EDBT) association}: Ch. Collet (2004 - 2013) in charge of  school organization program

\item {\it Member of the IJCAI-09 advisory committee}: M.-C. Rousset (2009)

\item {\it Deputy director of the French Mexican Laboratory in Informatics and Automatic Control (LAFMIA, UMI 3175) (2008 - ) }:  G. Vargas-Solar

\item {\it Director of Labex PERSYVAL-lab  }:  M.-C. Rousset(2012- )

\item \emph{Membre du conseil d'administration du VLDB Endowment}:  S. Amer-Yahia ( 2010-2013)

\item \emph{Membre du conseil ex\'ecutif de ACM SIGMOD }:  S. Amer-Yahia (2010-2012)

\item \emph{Membre de IEEE, ACM (membre s\'enior) et ACM SIGMOD}:  S. Amer-Yahia

\end{itemize}

%. - . -. - . -. - . -. - . -. - . -. - . -. - . -. - . -. - . -. - . -. - . -. - . -. - . -. - . -. - . -. - . -. - . -. - . -. - . -. - . -. - . -
 %\subsubsection*{Administration of Professional Societies}
%. - . -. - . -. - . -. - . -. - . -. - . -. - . -. - . -. - . -. - . -. - . -. - . -. - . -. - . -. - . -. - . -. - . -. - . -. - . -. - . -. - . -
%. - . -. - . -. - . -. - . -. - . -. - . -. - . -. - . -. - . -. - . -. - . -. - . -. - . -. - . -. - . -. - . -. - . -. - . -. - . -. - . -. - . -
\subsubsection*{Editorial Boards}
%. - . -. - . -. - . -. - . -. - . -. - . -. - . -. - . -. - . -. - . -. - . -. - . -. - . -. - . -. - . -. - . -. - . -. - . -. - . -. - . -. - . -

\begin{itemize}
\setlength{\itemindent}{-0.5cm}
\setlength{\itemsep}{-0.1cm}
\item {\it PVLDB, publication of the Very Large Database Endowment:} Ch. Collet (2008-2011)

\item {\it Computacion y sistemas}: G. Vargas-Solar, since 2002

\item {\it ICDIM Journal special issue}: G. Vargas-Solar, since 2005

\item {\it KER Journal special issue}: G. Vargas-Solar, since 2007

\item {\it ActaPress Journal}: G. Vargas-Solar, since 2008

\item {\it Interstices}: M.-C. Rousset

\item {\it ACM Transactions on Internet Technology (TOIT)}: M.-C. Rousset,  until 2005

\item {\it AI Communications( AICOM )}: M.-C. Rousset

\item {\it Communications of the ACM }: M.-C. Rousset,  since 2009

\item {\it advisory board of 21th International Joint Conference on Artificial Intelligence (IJCAI-09)}: M.-C. Rousset,  since 2009

\item \emph{R\'edacteur en chef de ACM TODS, Jan. 2011-2014}:  S. Amer-Yahia

\item \emph{R\'edacteur en chef de VLDB Journal, Sep. 2009-2015}:  S. Amer-Yahia

\item \emph{R\'edacteur en chef de Information Systems Journal (social search and recommendation and user influence in social media) depuis Juin 2010}:  S. Amer-Yahia
    
%\item \emph{R\'edacteur en chef de IEEE Data Engineering Bulletin, 2008-2009}:  S. Amer-Yahia

%\item \emph{R\'edacteur en chef de Encyclopedia of Database Systems, XML, Springer, 2008}:  S. Amer-Yahia

\end{itemize}


%. - . -. - . -. - . -. - . -. - . -. - . -. - . -. - . -. - . -. - . -. - . -. - . -. - . -. - . -. - . -. - . -. - . -. - . -. - . -. - . -. - . -
\subsubsection*{Chair  of Conferences and Workshops}
%. - . -. - . -. - . -. - . -. - . -. - . -. - . -. - . -. - . -. - . -. - . -. - . -. - . -. - . -. - . -. - . -. - . -. - . -. - . -. - . -. - . -

\begin{itemize}
\setlength{\itemindent}{-0.5cm}
\setlength{\itemsep}{-0.1cm}

\item \emph{Conference PC Chair}: SIGMOD Industrial 2015, BDA 2015, EDBT 2014, CIKM 2008,  S. Amer-Yahia;  \\ 
Conference SIIE'2012 (4th internationale conference  on information systems and economical Intelligence ). M.-C. Rousset 

\item \emph{ Chair of 5th International Conference on Information Systems and Economic Intelligence (SIIE 2012)}: M.-C. Rousset


\item \emph{Track Chair}: PVLDB 2013, SIGIR 2012, SIGMOD 2011, ICDE 2010, WWW 2010, ICDE 2008, S. Amer-Yahia

\item \emph{Industrial Chair}: EDBT/ICDT 2012, VLDB 2009,  S. Amer-Yahia

\item \emph{Demonstration Chair}: EDBT 2011, S. Amer-Yahia

\item \emph{Workshop Chair}: PersDB 2009, XSym 2006, WebDB 2004, S. Amer-Yahia

\item \emph{Parallel Data Mining Workshop}, in conjunction with Conference SIAM Data Mining 2011,  Mesa, USA: Alexandre Termier  

\item \emph{Tutorial Chair}: SIGMOD 2009, S. Amer-Yahia ; Tutorial chair of International Joint Conference on Artificial Intelligence 2011 (IJCAI),  Barcelona,  M.-C. Rousset ; 

\item \emph{ Tutorial Chair of IJCAI-11 }: M.-C. Rousset

\item \emph{ Area Chair of IJCAI-13 }: M.-C. Rousset
 
\end{itemize}
%. - . -. - . -. - . -. - . -. - . -. - . -. - . -. - . -. - . -. - . -. - . -. - . -. - . -. - . -. - . -. - . -. - . -. - . -. - . -. - . -. - . -
\subsubsection*{Organization of Conferences and Workshops}
%. - . -. - . -. - . -. - . -. - . -. - . -. - . -. - . -. - . -. - . -. - . -. - . -. - . -. - . -. - . -. - . -. - . -. - . -. - . -. - . -. - . -

\begin{itemize}
\setlength{\itemindent}{-0.5cm}
\setlength{\itemsep}{-0.1cm}

\item \emph{Extended Data Base Technology (EDBT) school 2009}, Ch. Collet, T. Delot and G. Vargas-Solar

\item \emph{Extended Data Base Technology (EDBT) school 2013}, S.  Amer-Yahia, Ch. Collet and G. Vargas-Solar

\item \emph{Conf{\'e}rence Gestion de Donn{\'e}es, principes, Technologies et Applications (BDA) 2014}, Ch. Bobineau et F. Jouanot
 
 \item \emph{Tutoriel Parallel Data Mining on Multicores}, International Joint Conference on Artificial Intelligence  2011 (IJCAI): Alexandre Termier

\end{itemize}

%. - . -. - . -. - . -. - . -. - . -. - . -. - . -. - . -. - . -. - . -. - . -. - . -. - . -. - . -. - . -. - . -. - . -. - . -. - . -. - . -. - . -
\subsubsection*{Program committee members}
%. - . -. - . -. - . -. - . -. - . -. - . -. - . -. - . -. - . -. - . -. - . -. - . -. - . -. - . -. - . -. - . -. - . -. - . -. - . -. - . -. - . -

\begin{itemize}

\setlength{\itemindent}{-0.5cm}
\setlength{\itemsep}{-0.1cm}
% \item {\it Conference or journal name}, participant name, year

\item {\it International Conference on Distributed Computing Systems (ICDCS)}: Ch. Collet (2007, 2009)

\item {\it Extended Data Base Technologies (EDBT)}:  Ch. Collet (2009, 2011, 2012 et 2013 (PHD workshop)), N. Ibrahim (2014), V. Leroy (2014)

\item {\it International Conference on Data Engineering (ICDE)}:  Ch. Collet (2009)

\item {\it IInternational Conference on Web Engineering (ICWE)}:  Ch. Collet (2010)

\item {\it International Conference on Model \& Data Engineering (MEDI)}:  Ch. Collet (2011)

\item {\it International Conference on World Wide Web (WWW)}:  Ch. Collet (2012 Web Engineering track), M.-C. Rousset (2012)

\item {\it International Conference on Web Information Systems and Technologies (WEBIST)}:  Ch. Collet ( 2012, 2013 et 2014), F. Jouanot (2012)

\item {\it International Conference on Cloud Computing and Services Science (CLOSER)}:  Ch. Collet ( 2012, 2013, 2014)

\item {\it  International Workshop on Information Management in Mobile Applications (IMMOA)}:  Ch. Collet ( 2012 et 2013)

\item {\it International Conference on Data Technologies and Applications (DATA)}:  Ch. Collet (2012, 2013 et 2014)

\item {\it International Conference on Information and Knowledge Management (CIKM)}: V. Leroy (2013)

\item {\it International Conference on Data Mining (ICDM)}: A. Termier (2009 ... 2013)

\item {\it SIAM International Conference on Data Mining (SDM)}: A. Termier (2009)

\item {\it International workshop on ambient data integration (ADI)}: F. Jouanot (2009)

\item {\it International Workshop on Data and Services Management in Mobile Environments (D2SME)}: Ch. Bobineau (2009)

\item {\it International ACM Conference on Management of Emergent Digital EcoSystems (MEDES)}, G. Vargas-Solar (2009)

\item {\it International Conference on the Applications of Digital Information and Web Technologies (ICADIWT)}: G. Vargas-Solar (2009)

\item {\it International Conference on Parallel and Distributed Systems  track on "Web Services" (ICPADS)}: G. Vargas-Solar (2009)

\item {\it IEEE IFIP Conference on e-Business, e-Services, e-Society}: Ch. Bobineau (2009)

\item {\it Database Engineering and Applications Symposium (IDEAS)}: CH. Collet (2010), Ch. Bobineau (2011, 2012, 2014), F. Jouanot (2010)

\item {\it International Conference on Tools with Artificial Intelligence (ICTAI)}: F. Jouanot (2010)

\item {\it Journal d'Ing{\'e}nierie des Syst{\`e}mes d'Information (ISI)}: Ch. Bobineau (2009), F. Jouanot (2010)

\item {\it Journal of ACM Transactions on Computer Systems}: Ch. Bobineau (2010)

\item {\it Journal Technique et Science Informatiques (TSI)}: Ch. Bobineau (2014)

\item {\it International Conference on Web Engineering (ICWE)}: F. Jouanot (2010)

\item {\it International Symposium on Wearable Computers (ISWC)}: F. Jouanot (2011)

\item {\it International Extended Semantic Web Conference (ESWC)}: F. Jouanot (2014)

\item {\it Very Large Data Bases Conference (VLDB)}: Ch. Collet (2011), M.-C. Rousset (2013)

\item {\it International Semantic Web Conference (ISWC)}: M.-C. Rousset (2011)

\item {\it International Conference on Database Theory (ICDT)}:M.-C. Rousset (2011)

\item {\it American Artifical Intelligence Conference (AAAI)}: M.-C. Rousset (2010)

\item {\it European Conference on Artifical Intelligence (ECAI)}: M.-C. Rousset (2010, 2014)

\item {\it International Joint Conference on Artificial Intelligence (IJCAI)}: M.-C. Rousset (2009, 2013), A. Termier (2013)

\item {\it European Semantic Web Conference (ESWC)}: M.-C. Rousset (2009, 2014)

\item {\it Reasoning Web Summer School}: M.-C. Rousset (2009)

\item {\it Bases de Donn{\'e}es Avanc{\'e}es (BDA)}:  Ch. Collet (2010, 2012, 2013), Ch. Bobineau (2009, 2011), G. Vargas-Solar(2011)

\item {\it Gestion des Donn{\'e}es dans les Syst{\`e}mes d'Information Pervasifs (GEDSIP)}: Ch. Bobineau (2009, 2010), G. Vargas-Solar (2009)

\item {\it Conf{\'e}rence en Recherche d'Information et Applications (CORIA)}: V. Leroy (2014)

\item {\it Conf{\'e}rence Extraction et Gestion de connaissances EGC'13}:  A. Termier (2009-2013) 

\item {\it International Conference on Computational Systems (ICCS)}: M.-C. Rousset (2014)
\end{itemize}

%%. - . -. - . -. - . -. - . -. - . -. - . -. - . -. - . -. - . -. - . -. - . -. - . -. - . -. - . -. - . -. - . -. - . -. - . -. - . -. - . -. - . -
\subsubsection*{Evaluation committee members}
%%. - . -. - . -. - . -. - . -. - . -. - . -. - . -. - . -. - . -. - . -. - . -. - . -. - . -. - . -. - . -. - . -. - . -. - . -. - . -. - . -. - . -
%
%\subsubsection*{International expertise}
%. - . -. - . -. - . -. - . -. - . -. - . -. - . -. - . -. - . -. - . -. - . -. - . -. - . -. - . -. - . -. - . -. - . -. - . -. - . -. - . -. - . -
%\subsubsection*{National expertise}
%. - . -. - . -. - . -. - . -. - . -. - . -. - . -. - . -. - . -. - . -. - . -. - . -. - . -. - . -. - . -. - . -. - . -. - . -. - . -. - . -. - . -

\begin{itemize}
\setlength{\itemindent}{-0.5cm}
\setlength{\itemsep}{-0.1cm}

\item {\it Vice-presidence  comit{\'e} d'{\'e}valuation scientifique  "Big Data, d{\'e}cision, simulation, HPC"}   Ch. Collet (2014-).

\item {\it ANR, Comite  de pilotage Mod{\`e}les num{\'e}riques}, Ch. Collet, 2010-2013.

\item {\it Member of the Specialists Council of the Delegation of Science and Technologies in Puebla, Mexico:}, G. Vargas-Solar(2007-).

\item {\it Member of the executive board of the National Network of Information and Communications Technologies, Mexico:}, G. Vargas-Solar (2010-).

\item {\it Invited member of the Comit{\'e} de S{\'e}lection of INSA de Lyon}: Ch. Bobineau (2009, 2010, 2013).

\item {\it Invited member of the Comit{\'e} de S{\'e}lection of Universit{\'e} de Valenciennes et du Haut Cambraisis}: Ch. Bobineau (2009, 2010).

\item {\it Invited member of the Comit{\'e} de S\'election of Universit{\'e} d'Aix-Marseille 3}: Ch. Bobineau (2011).

\item {\it Invited member of the Comit{\'e} de S\'election of Universit{\'e} de Bordeaux}: Ch. Collet (2014).

\item {\it Invited member of the Comit\'e de S\'election of Universit\'e Paris Nord}: A. Termier (2010, 2013).

\item {\it Invited member of the Comit\'e de S\'election of Universit\'e de Rennes 1}: A. Termier (2012).

\item {\it Member of the Comit\'e de S\'election of Universit\'e Joseph Fourier}: A. Termier (2010).

\end{itemize}



\subsubsection{Public Dissemination}

\subsubsection*{Panels}

\begin{itemize}

\item \emph{Social Sites: Challenges and Opportunities}, Georgia Koutrika (moderator), Amr Al Abbadi (UCSB), Sihem Amer-Yahia, Laks Lakshmanan
  (UBC), Raghu Ramakrishnan (Yahoo!Labs), In PersDB, Seattle, Sep. 2011.

\item \emph{Does Social Media Make News Generation and Consumption Better?}, Sihem Amer-Yahia (moderator), Krishna Gummadi (MPI), Jimmy Lin
  (U. Maryland and Twitter), Gilad Lotan (SocialFlow), Marcus Mabry (NY Times and International Herald Tribune), Mor Naaman (Rutgers
  U.), Catherine Quayle (PBS Need to Know). In the International  Conference on Web Logs and Social Media (ICWSM), July 2011.

%\item \emph{Crowds, Clouds, and Algorithms: Exploring the Human Side  of Big Data and Applications}, Michael J. Franklin (UC Berkeley and  Truviso, Inc.) - moderator, Sihem Amer-Yahia, AnHai Doan (U. Wisconsin and Kosmix), Jon Kleinberg (Cornell U.), Nick Koudas (U. Toronto and Sysomos, Inc.)  In ACM SIGMOD Conference, Indiana, IN, June 2010.

% \item \emph{The Big Players and Integrating Social Media}, Eugene  Agichtein (moderator), Sihem Amer-Yahia, Jeremy Hylton (Google), Matt Hurst (Microsoft). In the International Workshop on Search in Social Media (SSM), New York, NY, Feb. 2010.

%\item \emph{Does the Internet Make Us More Intelligent or More Stoopid?},  Serge Abiteboul (moderator), Sihem Amer-Yahia, Vassilis Christophides, Fabrice Le Fressant,  Manuel Serrano, Agn\`es Voisard. In the First INRIA Alumni workshop, Paris, France, Nov. 2009.

\item \emph{Masses de Donn{\'e}es, Big data et Data Management in Cloud}, Ch. Collet, M. Bouzeghoub, A. Laurent, D. Gros-Amblard. Conseil scientifique de l'INS2i; Feb. 2012. 

\end{itemize}

\subsubsection*{Keynotes}

\begin{itemize}

\item \emph{ Big Data and Smartgrids}, journ{\'e}e th{\'e}matique centr{\'e}e sur le traitement et la valorisation des donn{\'e}es appliqu{\'e}es au domaine de l'{\'e}nergie, Minalagic et Tenerrdis. Avril 2014, Ch. Collet.

\item \emph{D\'efis du Big Social Data Management}. Les Fondamentales du CNRS. Nov. 2013, S. Amer-Yahia.

\item \emph{User Activity Analytics on the Social Web of News}. 18th International Conference on Management of Data, COMAD, Dec 2012, S. Amer-Yahia.

\item \emph{Crowd-Sourcing Literature Review in SUNFLOWER}. 1st International Workshop on Crowdsourcing Web Search (CrowdSearch) in conjunction with WWW, Apr. 2012, S. Amer-Yahia.

\item \emph{User and Topic Analytics of the Social Web of News}. 5th International Conference on Information Systems and Economic
  Intelligence (SIIE), Feb. 2012, S. Amer-Yahia.
  
\item \emph{I am structured: Cluster Me, Don't Just Rank me}. 2nd International Workshop on Business  intelligencE and the WEB (BEWEB) in conjunction with EDBT, Mar. 2011, S. Amer-Yahia.

\item \em{Parallel Data Analysis}, Dagstuhl Seminar, 2013, A. Termier.

% \item \emph{Data Management for the Masses}. Workshop on Large Scale Data Processing (T\'el\'ecom ParisTech), Nov. 2010, S. Amer-Yahia

% \item \emph{Composite Retrieval of Stars and Chains}. Bases de Donn\'ees Avanc\'ees (BDA), Oct. 2010, S. Amer-Yahia

% \item \emph{Social Content Distribution and Recommendation}. International Workshop on Social Networks and Distributed Systems (SNDS) in conjunction with PODC, July 2010, S. Amer-Yahia

% \item \emph{I'll Have What She's Having: Recommendations in Social Content Sites}. Yahoo! Sponsored Seminar and Women in Computer Science at UC Irvine, Nov. 2009, S. Amer-Yahia

% \item \emph{Information Presentation of Ranked Datasets}. 10$^{th}$ Anniversary celebration of the W3C office in Greece, Sep. 2009, S. Amer-Yahia

% \item \emph{Information Presentation in Social Content Sites}. In the  SIGMOD PhD Workshop (IDAR), June 2009, S. Amer-Yahia

% \item \emph{Jelly: A Language for Building Community-Centric Information Exploration Applications} With Cong Yu. In the International Workshop on Modeling, Managing and Mining of Evolving Social Networks (M3SN), Apr. 2009, S. Amer-Yahia

% \item \emph{Information Exploration in Social Content Sites}. With Cong Yu. In the  International Workshop on Database Technologies for Handling XML Information on the Web (DataX), Mar. 2009, S. Amer-Yahia

\item \emph{Addressing Data Management on the Cloud: Tackling the Big Data Challenges}  at CONIELECOMP 2013, Puebla, Mexique, G. Vargas-Solar

\item \emph{Addressing Data Management on the Cloud: Tackling the Big Data Challenges} at ITESM 2013, Academic Leaders Seminars,  Puebla, Mexique, G. Vargas-Solar

\item \emph{ Addressing Data Management on the Cloud: Tackling the Big Data Challenges}, Puebla, at Alberto Mendelzon Workshop 2013 Mexique, G. Vargas-Solar

\item \emph{ Building data management services in clouds}, at Microsoft LATAM Faculty Summit 2012, Riviera Maya, Mexique, G. Vargas-Solar	

\item \emph{Reasoning on Web Data Semantics}, at Coll{\`e}ge de France, 2012, M.-C. Rousset

\item \emph{Cloud and Data Management: Research and Technical Challenges}, Conference UBIMOB 2010, Lyon, G. Vargas-Solar

\item \emph{Building the future of LATAM information and communication technologies}, at Microsoft LATAM Faculty Summit 2010, Guaruja, Brazil, G. 
Vargas-Solar

\item \emph{ ECLOUDSS : Building E-government Clouds using Distributed Semantic Services}. In Microsoft Research Summit, Buenos Aires, Argentina, 2009, G. Vargas-Solar

\item \emph{Services in Clouds : a new perspective for accessing the digital world}. In Conference IETI, Universidad Popular Autonoma de Puebla, 2009, G. Vargas-Solar


\item \emph{Observing the environment for building todays and future information systems}, Universidad Juarez Autonoma de Tabasco (2009), G. Vargas-Solar

\item \emph{Semana de Juarez, Mujer serpiente: une historia alternativa sobre el origen de la civilizacion}, Universidad Juarez Autonoma de Tabasco (2009), G. Vargas-Solar

\item \emph{Semantic oriented data spaces}. In Invited tutorial at EDBT Summer School on Data and Resource Management in Ambient Computing, Sept. 2009, M.-C. Rousset

\item \emph{Polyglot persistence and multi-cloud data management solutions}. In Invited tutorial at EDBT Summer School on Data all around
Big, Linked, Open,, Sept. 2013, G. Vargas-Solar 

%% tutorial Geno 

\end{itemize}



% - Collaborations internationales,
%With grants and/or Joint publications\\
%. - . -. - . -. - . -. - . -. - . -. - . -. - . -. - . -. - . -. - . -. - . -. - . -. - . -. - . -. - . -. - . -. - . -. - . -. - . -. - . -. - . -
\subsubsection{Principal International collaborations}
%. - . -. - . -. - . -. - . -. - . -. - . -. - . -. - . -. - . -. - . -. - . -. - . -. - . -. - . -. - . -. - . -. - . -. - . -. - . -. - . -. - . -

\begin{description}

\item[China:] The cooperation of HADAS with China started with the Beijing Institute of Automation (Chinese Academy of Science), more specifically the LIAMA French-Chinese laboratory, in 2007. We have obtained an ANR funding for the UBIQUEST project (Programme BLANC 2009-2012) where we have developed Case-Based Reasoning approach for optimizing distributed queries when (almost) no metadata are available. These techniques transparently exploit user-defined rule-based programs to combine network and data management.\\
The FP7 People CASES project extends this collaboration to the SouthEast University and the Aerospacial Center, both from Nanjing. The topics addressed in this context are service coordination data querying on the  Cloud applied to energy consumption control in industrial processes. Within these projects, senior and young researchers have done internships in France and China.

\item[Japon:] 
There are strong ties between A. Termier and Takashi Washio of Osaka University. Since 2011, they have been collaborating with the PhD student C. Kamdem-Kengne and her supervisors N. Ibrahim and M.-C. Rousset on combining optimization methods and pattern mining algorithms in order to find new ways of rewriting execution traces.

\item[Mexico:] the group has a long tradition (20 years) in developing cooperation actions between the Mexican and French governments in TICS. 
The cooperation of HADAS with Mexico includes the most important private and public institutions of that country: three major public research centres CINVESTAV, CICESE, INAOE, private centres like LANIA; and important universities like UDLAP, UATx, ITESM. 
The main research topics in the cooperation are services based infrastructures for managing distributed data with non-functional properties, services based query processing and flexible data storage services.  Collaboration on these topics has been formalized through projects (see below) and PhD students: A. Portila-Flores \footnote{Double diploma funded by the PROMEP Mexican program and the Foundation Jenkins.}, V. Cuevas \footnote{Funded by the project ANR - ARPEGE OPTIMACS.}, J. Espinosa-Oviedo \footnote{Funded through a CONACyT fellowship in the context of the ECOS-ANUIES project ORCHESTRA.}, Carlos-Manuel Lopez-Enriquez \footnote{Funded by the project ANR OPTIMACS, the CONACyT and the Jenkins Foundation} and Juan Carlos Castrejon \footnote{Funded by an excellence fellowship of the doctoral school MSTII.}.  Ch. Collet, and G. Vargas-Solar co-advised these students and with Ch. Bobineau they also advised master students of Mexican institutions. Some of them continued their education as PhD. students in France: Lourdes A. Martinez Medina \footnote{Funded by the ANR project UBIQUEST.}. 

Since 2008, G. Vargas-Solar is deputy director of the French Mexican Laboratory in Informatics and Automatic Control (LAFMIA, UMI 3175 http://lafmia.imag.fr) an international unit of the CNRS. 
The cooperation of HADAS with Mexico and the LAFMIA has lead to scientific results and to the education of graduate students through co-advising contracts and the organisation of thematic schools. 

\item[Vietnam:] We have developed since 15 years a strong collaboration with the Hanoi University of Science and Technology, including the International Research Institute MICA (UMI CNRS), and the National Universities in Danang and Ho-Chi-Minh City (4 doctors formed). This cooperation concerns adaptable distributed query optimization over stored data and data streams. Collaboration actions concern visits of senior scientists (2-3 times per year) and co-supervision of students.

\item[Brazil:] 
The collaboration with Brazil includes mainly the Universite Federal Rio Grande do Norte, department DIMAP,  equipe FORALL since 2007. This collaboration concerns service based data management on the cloud: semantic integration of data services through mashups, policy based service coordination, data processing on the cloud using map-reduce models, query langagues based on service compositions.  We priviledged applications on meteorology data and energy distribution. These commun research activities were funded by different organizations: 
Microsoft-LACCIR (e-CLOUDSS), the CNRS STICAMSUD program (CLEVER, SWANS) and by co-advising graduate students. In addition, postdoctoral and PhD and senior scientists internships in HADAS were funded by the CAPES and the CNRS.   

\item[Uruguay:] 
The collaboration with Uruguay concerns the University of La Republica since 2007. This collaboration concerns the specification of mashup languages including policies for associating quality of service attributes for retrieving and integrating data. The collaboration is done through commun projects (e-CLOUDSS, CLEVER, SWANS), co-advising of graduate students and invitations for lecturing seminars on topics related to the commun projects.

\item[Spain:] 
The collaboration with Spain concerns the region of Madrid and includes Universidad Rey Juan Carlos and Universidad Politecnica de Madrid. The collaboration concerns the specification of service oriented methodologies including non-functional properties. Collaboration actions concern visits of senior scientists, participation in PhD. viva, development of plug-ins for building a framework implementing the methodology $\pi$-SODM that we have proposed in the PhD of Placido Souza Neto. 

\end{description}


\subsection{Social, economical, and cultural impact} % (fold)
\label{sub:hadas_social_economical_and_cultural_impact}
% C3 : Impact social, e conomique et culturel
% 1 page
% - Main contracts and grants,

Results of our research have direct impact on applications dealing with huge amounts of data and resources in pervasive environments. 
They include traditional enterprise applications such as mining logs and traces, web applications but also "e-science" applications (in astronomy, biology, earth science, smartgrids, etc.). 
Environments we consider are wireless sensor networks (e.g. natural environment surveillance, industrial process monitoring), peer-to-peer data sharing, application deployment and maintenance for smart grids, transports, networks, smart cities.  

% - Main contracts and grants,
 %. - . -. - . -. - . -. - . -. - . -. - . -. - . -. - . -. - . -. - . -. - . -. - . -. - . -. - . -. - . -. - . -. - . -. - . -. - . -. - . -. - . -
\subsubsection{Main Contracts and grants}
%. - . -. - . -. - . -. - . -. - . -. - . -. - . -. - . -. - . -. - . -. - . -. - . -. - . -. - . -. - . -. - . -. - . -. - . -. - . -. - . -. - . -

%\subsubsection{External contracts and grants (Industry, European, National)}

\begin{description}

\item[Webcontent] (RNTL, The semantic web framework-2006-2009), 10 partners(CEA LIST, EADS DCS, Thales Research  $\&$ Technology, France Telecom R  $\&$ D, ADRIA D{\'e}veloppement, Soredab SAS, Exalead, New Phenix, Xyleme, INRIA-GEMO, INRA, INRIA-Mostrare, LIP6, PRISM, INRIA-InSitu, LIG, LIMSI-CNRS, GRIMM, EXMO, PSY-CO), \\
(http://www.webcontent-project.org/), 83300 \euro. Coordinator: CEA, Scientific lead in LIG: Ch. Collet and M.-C. Rousset. 
The WebContent project is creating a software platform to accommodate the tools necessary to efficiently exploit and extend the future of the Internet: the Semantic Web. The first targeted domain is the watch, a subpart of intelligence dedicated to warn the decider on the occurrence of an event or the evolution of a situation. It joins several Open Source tools to create the core of a Service Oriented Application and it defines the interface of several services that are available through several partners, either freely or through commercial licences. These services then exchange data in a formalized manner.

\item[OPTIMACS]: (ANR, program ARPEGE 2008-2011), 3 partners (LIG, LAMIH, LIRIS),  \\
(http://optimacs.imag.fr), 227 128 \euro. Coordinator: Grenoble INP-LIG, Scientific lead in LIG: G. Vargas-Solar. \\
OPTIMACS (SERVICE COMPOSITION BASED FRAMEWORK FOR OPTIMIZING QUERIES) combines hybrid query processing and services composition, addressing services composition and query processing including adaptive hybrid query optimization according to QoS criteria. OPTIMACS is an original research project that will lead to results with an important expected impact on ``modern data and services intensive systems" deployed on networks of heterogeneous devices, the so called ecosystem or dataspaces.

\item[DATARING] (ANR, Programme R{\'e}seaux du futur et services 2008-2011), 3 partners \\
(http://www.lina.univ-nantes.fr/projets/DataRing/). 130 549 \euro. Coordinator: INRIA Nantes, Scientific lead in LIG: M.-C. Rousset. \\
The DataRing project addresses the problem of P2P data sharing for online communities, by offering a high-level network ring across distributed data source owners. Users may be in high numbers and interested in different kinds of collaboration and sharing their knowledge, ideas, experiences, etc. Data sources can be in high numbers, fairly autonomous, i.e. locally owned and controlled, and highly heterogeneous with different semantics and structures. What we need then is new, decentralized data management techniques that scale up while addressing the autonomy, dynamic behavior and heterogeneity of both users and data sources.

\item[CONTINUUM] (ANR, Programme R{\'e}seaux du futur et services 2008-2011),  7 partners (I3S, LIG, SUEZ ENVIRONNEMENT, LYONNAISE DES EAUX, GEMALTO, LUDOTIC, MOBILEGOV), (http://continuum.unice.fr), 279 652 \euro. Coordinator: University of Nice, Scientific lead in LIG: F. Jouanot and M.-C. Rousset. \\
CONTINUUM (CONTINUITE DE SERVICE EN INFORMATIQUE UBIQUITAIRE ET MOBILE) addresses the problem of service continuity within the long-term vision of ambient intelligence. A core problem is to achieve software adaptation to a variety of resources in dynamic and heterogeneous environments with an appropriate balance between system autonomy and human control. Three key scientific issues will be addressed: context management and awareness, semantic heterogeneity, and human control versus system autonomy. The professions related to water management is used as a business application domain.

\item[UBIQUEST] (ANR, Programme BLANC 2009-2012), 3 partners (LIG, CITI, LIAMA), 149 099 \euro. Coordinator: Grenoble INP-LIG, Ch. Collet. Scientific lead in LIG: Ch. Bobineau and Ch. Collet. \\
UBIQUEST (Ubiquitous Quest: a declarative approach for integrated network and data management in wireless multi-hop networks) aims at integrating network and data management in dynamic ad-hoc networks. this integration will be done by giving a distributed database view of the whole network. Each node stores network and application data in a local database. Messages between nodes are queries or answers. The objective of this integration is the rapid development and deployment of applications and network protocols.

\item[Datalyse] (Investissement d'Avenir May 2013 -- November 2016), 219 950 \euro. \\
The aim of this project is to develop scalable algorithms for data mining and processing in collaboration with INRIA Saclay, LIFL, LIRMM and industrial partners: Eolas and B\&D. The project defines 3 use cases, all of them with industrial impact. The first use case, network analysis applied to data centers datasets, aims to provide interactive traffic monitoring interfaces including traffic aggregation over time abnormal traffic identification. The second use case, digital marketing, applied to server and application logs, aims to provide customer-centric statistics and customer engagement analysis using sequence mining. The third use case, linked open data, aims to develop a platform that integrates open data on the city of Grenoble and makes it readily available for the development of various applications.

\item[ALICIA] (ANR, February 2014 -- January 2017), 90 400 \euro. \\
The target of this project is the development of methods for information access and intelligent crowdsourcing in collaboration with Universit{\'e} Paris Sud, LTCI, Xerox, and UPS/IMT. In the context of information access (e.g. search or recommendation), building and maintaining user preference profiles helps applications satisfy diverse preferences. For intelligent crowdsourcing (e.g. data sourcing and micro-task completion), expertise profiles help better assign task to users. In both scenarios, the key challenges are that user preferences and expertise cannot be known in advance; and can rarely be explicitly declared by uses in a reliable or stable way. Consequently, preferences and expertise need to be discovered over time via a learning approach. Our project’s goal is the study of models and algorithms that rely on adaptive learning techniques to improve the effectiveness, performance, and scalability of user-centric applications.

\item[PAGODA] (ANR 12 JS02 007 01, Programme JCJC, January 2013 -- December 2016, coordinated by Meghyn Bienvenu), funding for one year post-doctoral researcher. \\
The aim of this project is to develop practical algorithms for ontology-based data access in collaboration with LRI, LIRMM, and the LADAF (Laboratoire d'anatomie de Grenoble). This project is centered on two challenges:
\textit{(i)} Scalability: in contrast with relational database management systems that benefit from decades of research on querying algorithms and optimizations, ontology-based data access is a young area of study, and despite important recent advances, including the identification of interesting tractable ontology languages, much work remains to be done in designing scalable OBDA query answering algorithms.
\textit{(ii)} Handling data inconsistencies: In real-world applications involving large amounts of data or multiple data sources, it is very likely that the data will be inconsistent with the ontology, rendering standard querying algorithms useless (as everything is entailed from a contradiction). Appropriate mechanisms for dealing with inconsistent data are thus crucial to the successful use of OBDA in practice, yet have been little explored thus far.

\item[SoGrid]  (2013-2017) - ADEME Le reseau  electrique de demain, 270 000 \euro. Scientific lead in LIG: Ch. Collet. Funding for one year post-doctoral researcher and one PHD. \\
SoGrid aims at confirming the path opened by ERDF in the technological revolution of the Smart Grid. The goal is to develop a global communication chain linking all components as the basis of future Smart Grids. By deploying intelligence all along this communication chain, SoGrid will allow (i) real-time supervision and control of the electrical grid; (ii) integration of decentralized renewable energy production; (iii) anticipation and support of new uses of electricity, in particular electric vehicles; (iv) the possibility to ensure that at every moment the  balance between production and consumption, particularly during peak consumption;  and (v) Control of consumption by the end-user and better quality of service. HADAS group is developing new optimized distributed and adaptable event stream management techniques taking into account the specificities of the Smart Grids.

\item[CLEVER] (2012-2013)  CLEVER CLOUD-BASED LATIN-AMERICAN ENVIRONMENTAL VIRTUAL OBSERVATORY. Scientific lead in LIG: Genoveva Vargas-Solar. \\
The project aims at providing the underlying services that will enable the VO to personalize and manage mashed up services. The result will be a platform where climate reports coming from different providers in LATAM will be mashed up. Resulting mashups will be exported as VO tools for eventually building other mashups.

\item[QUALINCA] (2012-2015). 80 600 \euro QUALINCA is a ANR Contint funded research project looking at developing mechanisms allowing to quantify the quality level of a bibliographical knowledge base, to improve the afore mentioned quality level, to maintain the quality when updating the knowledge base and to exploit the knowledge bases taking into account their quality levels. This project aims to develop mechanisms to: (i) describe the quality of an existing document database; (ii) maintain a given level of quality by controlling updates on such databases; (iii) improve the quality of a database; (iv) exploit these databases according to their level of quality.

\item[ SocTrace] (2011-2015)  FUI-Minalogic, OSEO. 374 700\euro. Partners: INRIA, LIG, TIMA, STMicroelectronics, Magilem, probayes. Coordinator: STMicroelectronics
Scientific lead in LIG: Alexandre Termier.  \\
The SoC-Trace project aims to develop a set of methods and tools based on traces of execution produced by multi-core embedded applications. It will allow developers to optimize and debug these applications more efficiently. Such methods and tools should become a building block for the design of embedded software, in response to the growing needs of analysis and debugging required by the industry. The technological barriers consist of a scaling problem (millions of events stored on gigabytes) and a trace understanding problem related to applications whose complexity is increasing. The project addresses the problem of controlling the volume of traces and of developing new analysis techniques. SocTrace is composed of academic partners with related themes, and several industry partners including STMicroelectronics.

\end{description}

%. - . -. - . -. - . -. - . -. - . -. - . -. - . -. - . -. - . -. - . -. - . -. - . -. - . -. - . -. - . -. - . -. - . -. - . -. - . -. - . -. - . -
\subsubsection{Research Networks (European, National, Regional, Local)}
%. - . -. - . -. - . -. - . -. - . -. - . -. - . -. - . -. - . -. - . -. - . -. - . -. - . -. - . -. - . -. - . -. - . -. - . -. - . -. - . -. - . -

\begin{description}

\item[E-CLOUDSS:] (BUILDING E-GOVERNMENT CLOUDS USING DISTRIBUTED SEMANTIC SERVICES, Microsoft, 2007-2011, LACCIR, http://e-cloudss.imag.fr), 5 partners (CNRS LIG-LAFMIA, Fundacion Universidad de las Am{\'e}ricas, Puebla, Mexique, Universidad de la Republica de Uruguay, Uruguay, Universidade Federal do Rio Grande do Norte), 50,000 USD. Coordinator: J.L. Zechinelli Martini, LAFMIA, Scientific lead in LIG: G. Vargas-Solar). 
The objective of E-CLOUDSS is to propose an infrastructure for mashing up reliable semantic services for building e-government clouds. Mashups represent a new wave for building Web applications. E-CLOUDSS addresses the management (definition and enforcing at execution time) of non functional properties associated to services coordination for building reliable mashups. Effective ways to perform virtual executions is one of the main subjects of study of E-CLOUDSS.

\item[WebIntelligence:] (Cluster R{\'e}gional "Informatique, Signal, Logiciels embarqu{\'e}s" - 2006-2009). The project aims at organizing research on web intelligence in Rhone-Alpes.  \\
suite WebIntelligence (ARC 6 et ARC 7) : Cloud Computing Research Group \\
(http://cloud.liris.cnrs.fr/wiki/doku.php). Our participation in the regional cluster addresses topics related to cloud computing, particularly big data collections integration and management on
multi-cloud environments guided by service level agreements. We work with the LIRIS lab and the Management School of University Lyon 3. Our commun research results are applied to smart energy, intelligent transport and political strategies cases. We actively, organize seminars and collective actions willing to encourage  research collaboration synergy among the actors of the region working on cloud computing and big data management.

\item[ORCHESTRA:] (ORCHESTRATION TRANSACTIONNELLE DE SERVICES, Program: ECOS-ANUIES 2007-2011), 3 partners (Grenoble INP, Universidad Autonoma de Tlaxcala, Fundacion UNiversidad de las Am{\'e}ricas, Puebla, Mexique). Coordinator: Ch. Collet. 
Missions for Professors (Ch. Collet in 2007 and  2008, and G. Vargas in 2009) and PhD students. The objective of ORCHESTRA is to propose an infrastructure pour building transactional, secure and evolutive service-based applications. The key elements of the project are: (i) the definition of a framework (general solution) of technical services for managing the security, transactional properties and evolution of business services ; and (ii) implementation of the framework an its validation in the development of service-based applications: production chains. 

\item[CASES] (2012-2015) - European Union FP7, PEOPLE program, UK, France, Ukrania, China. Customized Advisory Services for Energy-efficient Manufacturing Systems. Scientific lead in LIG: Genoveva Vargas-Solar.  156 000 \euro\\
The project aims at teaming up transcontinental researchers in the areas of sustainable manufacturing and information technologies to enrich the knowledge base and achieve research synergies to develop smart design and manufacturing services in terms of energy efficiency. The project  integrates the complementary expertise of the European, Chinese and Ukrainian teams to devise ICT-based smart services and standards to address the multi-faceted requirements of global eco-design and sustainable manufacturing planning.

\end{description}
%% LIA 

%. - . -. - . -. - . -. - . -. - . -. - . -. - . -. - . -. - . -. - . -. - . -. - . -. - . -. - . -. - . -. - . -. - . -. - . -. - . -. - . -. - . -
\subsubsection{Internal Funding}
%. - . -. - . -. - . -. - . -. - . -. - . -. - . -. - . -. - . -. - . -. - . -. - . -. - . -. - . -. - . -. - . -. - . -. - . -. - . -. - . -. - . -
% \textit{(BQR, MSTIQ projects, IMAG projects, CORDI, INRIA Post Docs, ARC, any funding from INRIA, Grenoble INP, UJF \ldots)}

 \begin{description}

\item[RED-SHINE:] (RELIABLY AND SEMANTICALLY INTEGRATING WEB INFORMATION BY MASHING UP DATA SERVICES, BQR Grenoble INP, 2009). 2 partners (LIG, LAFMIA-UMI 3175)  (http://lafmia.weebly.com/), 20 000\euro - one PhD grant and 4 months for inviting professors. Coordinator: Grenoble INP-LIG, Scientific lead in LIG: G. Vargas-Solar). \\
The objective of RED-SHINE is to propose an infrastructure for mashing up services using semantics and thereby integrating information from the Web. RED-SHINE will redefine and extend OQLiST for declaratively defining reliable semantic mashups. RED-SHINE addresses the management (definition and enforcing at execution time) of non functional properties (NF-P) associated to services' coordination for building reliable mashups. The objective of our work will be to propose a language for orthogonally expressing NF-P and ensuring strategies, and to specify execution strategies for adding NF-P to mashups.\\

\item[DAMOCLES:] (MSTIQ project, 2009). 2 partners (LIG, TIMA), 15 000 \euro - one year postdoc. Coordinator: Grenoble INP-LIG, Scientific lead in LIG: A. Termier). \\
DAMOCLES (DAta Mining for On Chip Low Energy Systems) aims at developing data mining algorithms for analyzing memory accesses in System-on-Chip processors, in order to optimise data placement and thus reduce energy consumption.

\item[Smart Energy:] Grenoble INP (2012-2014), Participants: LIG, G-SCOP, G2ELab, GIPSA-Lab. \\
This projects aims at federating the scientific communities from Grenoble INP supported laboratories around the development of Smart Grid technologies.
The idea is to identify common interests among researchers to propose new research projects.

\item[WalT:] Grenoble INP and  University Joseph Fourrier, Programme AGIR (2013-2015) on Wireless Testbed. \\
This project, proposed by HADAS and DRAKKAR research groups, aims at developing an easy configurable testbed composed of embedded computing devices (Rapsberry Pi), sensors and network equipments (e.g. commutators, wireless communications). It will be exploited to test new networking protocols or distributed database techniques developed in both research groups.

\item[Big join:]  Grenoble INP and  University Joseph Fourrier, Programme AGIR (2013-2015) on Mod{\`e}les et algorithmes pour les jointures de Big Data sur Map-Reduce, Scientific lead in LIG: S. Amer Yahia). \\

	
\end{description}

% - Aspects interdisciplinaires
% - Vulgarisation

%==========================================================================
\subsection{Team Organization and life} % (fold)
\label{sub:hadas_team_organization_and_life}%==========================================================================

% C4 : Organisation et vie de l 'entite 
% 1/2 page
% - Seminaires
% - Vie scientifique de l 'equipe
% - Prise en compte des recommandations de la pre�?ce�?dente e�?valuation

Full group meeting are organized once every two weeks for discussions about research works, progress concerning research contracts and so on. These meetings are an opportunity for students or permanents to present their works, their difficulties, to exercise presentation of their papers and/or to present interesting research papers they have found.
In parallel and with the same frequency, we organize meeting for permanent staff to discuss management aspects.

Every PhD student has to participate to at least one summer school and to assist to at least one major conference during his PhD. They also have to prepare a poster presenting their work that will be presented in very short sessions (two-minutes madness).

Each researcher is responsible of his contracts and manages the associated budget as he/she wants. Every one is fully autonomous in his research work and operation. But major decisions concerning the group are still taken collegiately. Some responsibilities are distributed among permanent people:
\begin{itemize}
\setlength{\itemsep}{-0.1cm}
\item management of offices and keys: N. Ibrahim
\item management of computing equipment and representation of the group with the lab for these aspects: Ch. Bobineau
\item Website administration and representation of the group with the lab for these aspects : F. Jouanot.
\end{itemize}
~\\
\noindent Finally, three full group seminars off-campus have been organized to discuss major points within the group since 2009:
\begin{itemize}
\setlength{\itemsep}{-0.1cm}
\item January 14th and 15th 2009 at the "Pic de la Belle Etoile" hotel in Pinsot.
\item April 8th and 9th 2011 at the "Pic de la Belle Etoile" hotel in Pinsot.
\item October 22nd and 23rd 2012 at the "Centaure" center in Réaumont.
\end{itemize}
~\\
\noindent The last evaluation report on our group read as follows: 
\begin{it}
\begin{itemize}
\setlength{\itemsep}{-0.4cm}
\item Points {\`a} am{\'e}liorer et risques :\\
Si l'activit{\'e}  de publication en revue est en hausse, les efforts doivent encore se poursuivre pour amener les diff{\'e}rentes th{\'e}matiques {\`a} un niveau comparable. Le nombre de HDR est faible comparativement au nombre de doctorants.
Les probl{\'e}matiques abord{\'e}es dans l'{\'e}quipe n{\'e}cessitent une synergie avec les syst{\`e}mes r{\'e}partis, le parall{\'e}lisme et les r{\'e}seaux. M{\^e}me si des collaborations sont d{\'e}j{\`a}  {\'e}tablies sur ces th{\'e}matiques avec les {\'e}quipes du LIG ou {\`a} l'ext{\'e}rieur, cet effort doit {\^e}tre poursuivi si l'on veut v{\'e}ritablement contribuer au plus niveau.\\
\item Recommandations :\\
L'{\'e}quipe doit augmenter son nombre de HDR pour g{\'e}rer au mieux ses doctorants et les contrats en cours. La mise {\`a} disposition d'un ing{\'e}nieur permettrait {\`a} l'{\'e}quipe d'aller plus loin dans la r{\'e}alisation et la valorisation de ses productions logicielles.
\end{itemize}
\end{it}

\noindent Two remarks are in order concerning this report.
\begin{itemize} 
\setlength{\itemsep}{-0.4cm}
\item First, the number of journal publications continued to increase  with a better repartition between the research themes of the group. The collaboration with groups working on networks, distributed systems and parallel algorithms were strengthened. Common publications or projects attest this. \\
\item Second, two  HDR  have been defended during this period in 2013 and 2014. The arrival of S. Amer-Yahia (DR CNRS) also helped in increasing the potential of coaching. 
Etienne Dubl{\'e}, research engineer part-time joint the group in 2011. He was a great help in developing prototypes. However, this  is not enough to have  a real valorisation strategy. 
\end{itemize}

% subsection team_organization_and_life (end)

% \newpage
%==========================================================================
\subsection{Training through research, educational involvement} % (fold)
\label{sub:hadas_training_through_research_educational_involvment}
%==========================================================================

% C5 : Implication dans la formation par la recherche
% 1/2 page
% - Nombre de the ses soutenues,
% - Suivi des doctorants,
% - devenir des doctorants
% - responsabilite s master ED...

\begin{description}

\item[Thesis:]  10 thesis have been defended;  On average, a thesis is done in four years. 
Also, two HDR have been defended. 
\begin{itemize}
\item \emph{Pattern mining rock: more faster, better.} Alexandre Termier, HDR UJF. 

\item \emph{Efficient, continuous and reliable Data Management by Coordinating Services.} Genoveva Vargas-Solar, HDR Grenoble INP. 
\end{itemize}

\item[PhD student supervision:] regular meetings, presentations of papers, "workshop" on talks / papers, participation to schools, poster.

\item[Future:]  
%Le devenir des doctorants est varie  : X dans l'industrie (Chercheur ou ingenieur R $\&$ D) ;  2 dans le superieur. 
The professional perspectives of students are diverse: 5 former PHD are in industry (researcher or engineer R  $\&$ D) and 5 have post-doc or academic positions. 

\end{description}

%. - . -. - . -. - . -. - . -. - . -. - . -. - . -. - . -. - . -. - . -. - . -. - . -. - . -. - . -. - . -. - . -. - . -. - . -. - . -. - . -. - . -
\subsubsection*{Educational involvement} % (fold)
%. - . -. - . -. - . -. - . -. - . -. - . -. - . -. - . -. - . -. - . -. - . -. - . -. - . -. - . -. - . -. - . -. - . -. - . -. - . -. - . -. - . -

% subsection training_through_research_educational_involvment (end)
Supervision of Educational Programs:
\begin{itemize}

\item M.-C. Rousset: Co-director for the Master of Sciences in Informatics (2009-2013)

\item N. Ibrahim: Co-director for the Master of Sciences in Informatics (2014 -) 

\item Ch. Bobineau: 
\begin{itemize}
\setlength{\itemsep}{-0.4cm}
\item Correspondant Vietnam pour Grenoble INP - Ensimag. \\
\item Charg{\'e} de projet PFIEV (Programme de Formation d'Ing\'enieurs d'Excellence au Vietnam) pour Grenoble INP
\end{itemize}
\end {itemize}

%==========================================================================
\subsection{Strategy and Research Project} % (fold)
\label{sub:hadas_strategy_and_research_project}
%==========================================================================

% Crite re C6 : Strategie et projet a  cinq ans
% 1 page
% - Projection pour les 5 ans a  venir

The research perspectives of HADAS  (Heterogeneous and adaptive data management systems)  are guided by  new challenges introduced by the continuous production of huge and heterogeneous data collections that require data management technologies to support  several activities  such as capturing, integrating, searching/querying, filtering, indexing, recording, annotating, etc. for analysis. 
However, before being analyzed data needs preparation, transformation, loading into specific data management tools, etc. It is clear that database infrastructures are the bottleneck for efficient data analysis. We propose to contribute to the construction of such an infrastructure that may maximize the activities of data analysis. 
Our perspective global view for the incoming five years is to continue our research in the general framework of putting data-centric ideas outside classical database systems, merging them with knowledge and
semantic descriptions of data and resources for a service-based infrastructure adapted to large, distributed and heterogeneous data sets. 
% reasoning in new architectural supports such as large-scale clusters, multi-core processors, ad-hoc networks. 

It is quite clear that the proposed solutions have to fulfill  properties such as efficiency, adaptivity, fault-tolerance, security, confidentiality, and privacy. 
HADAS research will address:
\begin{itemize}
\item {\bf\em Management of large datasets particularly focussing on}:
\begin{itemize}
	\item Adaptive and distributed persistency systems (storage/cache) for storing heterogeneous datasets.
	\item Indexing data on the fly.
	\item Economy and energy oriented integration of big datasets management : economic cost model .
\end{itemize}
\item  {\bf\em Adaptive query processing}
\begin{itemize}
	\item Declarative hybrid languages for expressing parallel data processing .
	\item Adaptive query operators : data is reorganized on-the-fly as part of the query operators, while future queries exploit and continuously enhance this knowledge.
	\item Learning-based distributed query optimization.
	\item Service Level Agreement  guided optimization of continuous and mobile queries.
	\item Quality-based continuous data/event stream processing and composition.
\end{itemize}
\end{itemize}

Sustainable mobility and urban systems like smart cities, energy, clean, safe and efficient technologies like Smart Grids, smart energy, clean technologies and  data markets for extracting business value from data, are examples of applications that call for an intelligent, adaptive, efficient and scalable data and knowledge management infrastructure. 

Our agenda falls within the  scientific research directions on Information and Communication Technologies given by  the H2020 program. We identified two of them for our group:
\begin{itemize}
\item Advanced Cloud Infrastructures and Services
\item Big Data Innovation and take-up and Big Data  
\end{itemize}
% -- Research for (i) Innovative data products and services and (ii) Solving fundamental research problems
% The idea is to define technologies combining big data, internet of things in the cloud. 
It also concerns the societal chalenges: Secure, clean and efficient energy. 

% subsection strategy_and_research_project (end)
This vision is nowadays well accepted as we have to considered large and heterogeneous data sets, huge numbers of connected devices with data management capabilities and increasing numbers of users/ applications. 


%==========================================================================
\subsection{Self assessment} % (fold)
\label{sub:hadas_self_assesment}
%==========================================================================

% 1 page
% - [SWOT] Strengths, Weaknesses, Opportunies, Threats

Our main strength is our experience for designing and developing components at the heart of data management systems. HADAS has been one of the pioneer groups proposing the unbundling of database systems and considering future database system as a large-scale semantic-based infrastructure for data and resources management. It raised many fundamental questions  which have been the basis of fundamental research published in top journal and conferences. 
The research has been done with PHD and post-doc. The  number of doctoral students is rather stable over the last five years. We have 10PhD per year, all supervised by faculty members, resulting in 2 PhD's defense per year. 
From the point view of research dissemination and recognition we have a very good impact with a lot of keynotes talks, tutorials and conferences
programs participation among which are the major conferences in the domain of data management. 

A strong positive point of our activity is the numerous and fruitful collaborations that we have. Several are internal to LIG: EXMO, AMA, ERODS, NANOSIM and MESCAL. This is attested by common publications or projects. Others collaborations are: (i)  local with our strong implication in the PERSYVAL-lab labex, (ii) national with our participation to several ANR projects, the "Investissement d'Avenir" program and the creation of the future GDR on Big Data/Data Sciences. 
At the international level we have numerous  collaborations.  We have strong relationships for years with Mexico (G. Vargas is the deputy director of the LAFMIA lab) that had lead to scientific results (publications and prototypes) and to the education of graduate students through co-advising contracts. We also have had collaborations with Japon and Vietnam. During this period we developed exchanges with Brazil, Uruguay (research network), Spain and and China(network CASE).  
%HADAS has a true international visibility.
From the point of views of contracts, the group is very active. We have an average of 350Keuros as inputs per year. This means around 50Keuros per (equivalent) permanent researcher per year. Contracts allow us to hire PhD students, post-doc and engineers. During the period we also got 3 grants from the french research Ministery, including an excellence grant from the UJF. 

% Ne : 5,83 + ing�nieur

Managing and developing collaborations, PhD supervision, contracts is time consuming and is a potential cause of difficulty. Despite the fact that we tried to have a management of the group taking care of no dispersion, we could not succeed to define a unique  scientific project for the incoming years. The positive aspect is the emergence of a new group (SLIDE) and the focusing of the HADAS group (Heterogeneous and Adaptive Data Management Systems) on research on data technologies combining big data, internet of things and  cloud computing. 
The Big data phenomenon is a real opportunity to validate our component-based data systems vision as we have to (i) consider large and heterogeneous data sets, huge numbers of connected devices with data management capabilities and increasing numbers of users/applications,  and (ii) face the challenges of scalability,  heterogeneity, distribution, efficiency, quality, security and adaptivity. 
The smart grids domain we choose to explore is very promising as the management of data (where to put the data, what to do with it, which data to collect,  integrate, summarize, how to access it efficiently, ... ) is the foundation for developing intelligent metering systems and adaptive supervisory control able to handle huge amount of events and alerts.  

The main problems of the group will still remain, namely: i) a lack of critical mass in terms of permanent staff even if we plan to hire two post-doc for the two incoming years.  These problems will probably hinder the activities of the group in terms of support for contractual work, and to a lesser extent in the supervision of doctoral students.


% subsection self_assesment (end)


% section equipe (end)


%%% Local Variables: 
%%% mode: latex
%%% TeX-master: "master"
%%% LaTeX-command: "pdflatex -shell-escape"
%%% End: 
