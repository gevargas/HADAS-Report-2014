% Pour inclure une illustration utiliser la macro suivante
% \dopdffig{ficher}{largeur}{legende}{label pour reference}
% exemple : \dopdffig{graph1.pdf}{.6\textwidth}{Resultats de l'annee}{fig:res_annee}

% {\'e} {\`e}
% {\`a}

\section{Equipe HADAS - Axe Traitement de Donn{\'e}es et de Connaissances {\`a} Grande Echelle} 
\index{HADAS}
\label{sec:hadas}

%==========================================================================
\subsection{Scientific Presentation} % (fold)
\label{sub:hadas_scientific_presentation}
%==========================================================================

% 1 page pour la description
%  Research group members
%  Group evolution in terms of members

% . .. .. .. .. .. .. .. .. .. .. .. .. .. .. .. .. .. .. .. .. .. .. .. .. .. .. .. .. .. .. .. .. .. .. .. .. .. .. .. .. .. .. .. .. .. .. .. .. .. .. .. .. . 
\paragraph{Research group:}
% . .. .. .. .. .. .. .. .. .. .. .. .. .. .. .. .. .. .. .. .. .. .. .. .. .. .. .. .. .. .. .. .. .. .. .. .. .. .. .. .. .. .. .. .. .. .. .. .. .. .. .. .. . 

The Hadas group was founded in october 2005 as a new team for the LIG laboratory. It actually follows the STORM team, directed by M. Adiba until 1996. 
Over the years, the group  proposed an evolution of the scientific vision of a database management systems  as  a semantic-based infrastructure for managing ubiquitous and heterogeneous data services. 
The team currently includes 9 research persons (2 full professors, 5 Associate Professor and two CNRS Research Scientist),  a research engineer and 15 PHD students and post-docs. During this period we hired 4 people. 

% \textit{(Reference date is Oct. 1st 2009-june 2014) 1 page}\\
% \textit{(including outgoing doctoral students who have not yet presented their thesis, even if listed as departing in OSE). }

%\begin{flushleft}
%\begin{minipage}{\linewidth}
%\renewcommand{\footnoterule}{} 
%\tablehead{}\begin{supertabular}{|m{3cm}|m{3cm}|m{3cm}|m{2.5cm}|m{2cm}|}
%\hline
%\multicolumn{5}{|c|}{\cellcolor{gris}\textit{Permanent Researchers}}\\\hline
%\cellcolor{grisclair}Name & \cellcolor{grisclair}First name & \cellcolor{grisclair}Function & \cellcolor{grisclair}Institution & \cellcolor{grisclair}Arrival date \\\hline
%Bobineau & Christophe & Associate Professor & Grenoble INP & Sep 2003 \\\hline
%Collet & Christine & Full Professor & Grenoble INP & Sep 1999 \\\hline
%Collet & Christine & Full Professor & UJF & Jan 91 \\\hline
%Jouanot & Fabrice & Associate Professor & UJF & Sep 2003 \\\hline
%Rousset & Marie-christine & Full Professor & UJF & Sep 2005 \\\hline
%Termier & Alexandre & Associate Professor & UJF & Sep 2007 \\\hline
%Vargas-solar & Genoveva & Research Scientist & CNRS & Jan 2002 \\\hline
%ajouter : Sihem et Vincent, Etienne
%\end{supertabular}
%\end{minipage}
%\end{flushleft}


% \textit{(Free form description of how the team composition has evolved: up to 15 lines)}\\
Research activities on services has been reinforced in the group with the arrival of Noha Ibrahim, associate professor who joined the team in October 2010. 
Then, the arrival of Sihem Amer-Yahia Research (Dec 2011) and Vincent Leroy Associate Professor UJF / Research scientist (Sept 2012) brought new areas of research in the group. 
Etienne Dubl{\'e} Research Engineer (Mai 2011) helps us in developing and finalizing some of our prototypes. He also brought expertises in new kind of sensor networks and of file systems. 


% Research description  themes 
% subsection scientific_presentation (end)
% . .. .. .. .. .. .. .. .. .. .. .. .. .. .. .. .. .. .. .. .. .. .. .. .. .. .. .. .. .. .. .. .. .. .. .. .. .. .. .. .. .. .. .. .. .. .. .. .. .. .. .. .. . 
\paragraph{Research description:}
%\
% . .. .. .. .. .. .. .. .. .. .. .. .. .. .. .. .. .. .. .. .. .. .. .. .. .. .. .. .. .. .. .. .. .. .. .. .. .. .. .. .. .. .. .. .. .. .. .. .. .. .. .. .. . 

The HADAS group has contributed in the following areas: relational data models, snapshots and their semantics, active and temporal databases and object-oriented database systems. The advent of the web and middleware infrastructures in the early 1990s has profoundly changed the nature of research in databases. 
Our research is related to the changes in devices and softwares:
\begin{itemize}
\item Memories and disks with more capacity and faster access
\item Faster processors and networks
\item Algorithmic advances, e.g. parallel computing
\item Cloud computing (virtualization, elasticity, pay-as-you-go ...)
\end{itemize}

More recently big data management and analysis introduces  more challenging perspectives. Technological changes have reduced the cost of creating, capturing, managing and storing information to a sixth of what it was in 2005. This has allowed a scale change in the size of data, distribution of data, number of connected devices, and number of users. 

Considering this globalization of data, knowledge and computing resources and given the quality and intelligence of the expected data management functions, we have been confronted (and are still confronted) with a change in the field of databases.

To face these challenges we decided to revisit database systems and consider them not anymore as centralized data storage systems but as data management services largely distributed and deployed over different types of large scale systems (grids, peer-to-peer networks, sensor networks, ambient and ubiquitous environments). 
Semantics is at the heart of this approach as it is used at all levels of the process of designing or composing data services for handling autonomy, dynamic behavior and heterogeneity of both users and data sources. 

The activities of the group during these past years have been centered on the following themes:
\begin{itemize}
\item  Accessing data in large-scale systems: a first aspect concerns query optimization in distributed and dynamic systems ; a second aspect deals with mining large amounts of data to extract patterns of interest.
\item  Composing data services in a dynamic way:  we investigate models, algorithms and tools for coordinating services with non functional properties (contracts) and for providing access to heterogeneous data coming from services
\item  Reasoning on data semantics: we investigate different models and  algorithms  for querying data (or resources) through  possibly heterogeneous and distributed ontologies.
\end{itemize}

We have participated to the Optimacs and Ubiquest ANR projects that bridge the gap between data management and (web) services and  between networks and data management.   

We have been also involved in other ANR projects on context management for software adaptation, handling uncertainty and trust in peer-to-peer data management systems. We are involved in the Datalyse project whose objective is to build a smart warehouse demonstrator for the collection, certification, integration, categorization, security, enrichment and sharing of heterogeneous Big Data  of type "Big Data User" (UBD) or from machines, of type "Big Data Monitoring" (MBD). 
We also collaborate with industry, specially with ST Microlectronics on novel data mining algorithms able to fully exploit the parallelism of multicore machines. 

Results of our research have direct impact on applications dealing with huge amounts of data and resources largely distributed in pervasive environments, such as data spaces, smart grids and smart buildings, hardware and software observation, and the semantic Web.

%==========================================================================
\subsection{Scientific and Technological Results} % (fold)
\label{sub:hadas_scientific_and_technological_results}
%==========================================================================
% Critere C1 : Qualite scientifique et production
% 2 pages
%  resultats scientifiques majeurs

%. - . -. - . -. - . -. - . -. - . -. - . -. - . -. - . -. - . -. - . -. - . -. - . -. - . -. - . -. - . -. - . -. - . -. - . -. - . -. - . -. - . -
\subsubsection{Accessing data in large-scale systems}
\label{optimisation}
%. - . -. - . -. - . -. - . -. - . -. - . -. - . -. - . -. - . -. - . -. - . -. - . -. - . -. - . -. - . -. - . -. - . -. - . -. - . -. - . -. - . -


Accessing data concerns several aspects of large scale systems: number of resources, data volume and data complexity. It basically means using declarative queries that are optimized based on system characteristics.
Data mining is another way to query large quantities of data, by extracting interesting patterns from them. Such patterns provide meaningful abstractions of raw data, which are thus less numerous and more appropriate for data analysis. 
Globally, the difficulty for evaluating queries efficiently on nowadays applications motivates this work to revisit traditional query optimization techniques. The following presents these two aspects of accessing data in the large. It also focuses on works done on querying the social web. 

%. - . -. - . -. - . -. - . -. - . -. - . -. - . -. - . -. - . -. - . -. - . -. - . -. - . -. - . -. - . -. - . -. - . -. - . -. - . -. - . -. - . -
\subsubsection*{1- CBR query optimization}
%. - . -. - . -. - . -. - . -. - . -. - . -. - . -. - . -. - . -. - . -. - . -. - . -. - . -. - . -. - . -. - . -. - . -. - . -. - . -. - . -. - . -


Our research contributes to the development of new distributed query optimization techniques. It relies on the adaptation of machine learning, more precisely Case-Based Reasoning(CBR), and pseudo random search space exploration (also exploiting the case base) to produce efficient  query execution plans according to application specific optimization objectives expressed over resource consumption (e.g. time, energy, number of messages).  The query plan generation considered multiple optimization objectives customizable to application requirements (QoS based Hybrid Query optimization).

These research led to the following original contributions:
\begin{itemize}
\item  A query optimization approach that use cases generated from the evaluation of similar past queries. A query case comprises: (i) the query (the problem), (ii) the query plan (the solution) and (iii) the measures of computational resources consumed during the query plan execution (the evaluation of the solution). 

\item  A query plan generation process [1] that uses classical query optimization heuristics and makes decisions randomly when information on data is not available (e.g. for ordering joins, selecting algorithms or choosing message exchange protocols). This process also exploits the CBR principle for generating plans for subqueries, thus accelerating the learning of new cases. 

\item  A Simulation Platform [2] allowing to experiment distributed query optimization and rule-based programs over a set of distributed data-enabled devices hosting virtual machines(VM). A VM integrates a query optimization engine [3] implementing the above techniques. 
\end{itemize}

\begin{description}
\item[Contracts:] ANR projects: UBIQUEST, OPTIMACS (coordinators);  AGIR project Wait; FP7 EU project Cases(coordinator).  

\item[Key references:]~% (if necessary)
%\cite{di:bbg+09} 

[1]	Lourdes Martinez, Christine Collet, Christophe Bobineau, Etienne Dubl{\'e}. The QOL approach for optimizing distributed queries without complete knowledge. IDEAS, 91-99, 2012

[2] Ahmad Ahmad-Kassem, Christophe Bobineau, Christine Collet, Etienne Dubl{\'e}, St{\'e}phane Grumbach, FudaMa, Lourdes Martinez, St{\'e}phane Ub{\'e}da. UBIQUEST, for rapid prototyping of networking applications. IDEAS, 187-192, 2012

[3] Lourdes Martinez, Christine Collet, Christophe Bobineau and Etienne Dubl{\'e}. CoBRa for optimizing global queries. BDA, 2013

%[4] Carlos-Manuel Lopez-Enriquez, Genoveva Vargas-Solar, Jos{\'e}-Luis Zechinelli-Martini, Christine Collet. Hybrid query generation,  LANMR,  117-128, 2012

\end{description}

%. - . -. - . -. - . -. - . -. - . -. - . -. - . -. - . -. - . -. - . -. - . -. - . -. - . -. - . -. - . -. - . -. - . -. - . -. - . -. - . -. - . -
\subsubsection*{2- Data mining}
%. - . -. - . -. - . -. - . -. - . -. - . -. - . -. - . -. - . -. - . -. - . -. - . -. - . -. - . -. - . -. - . -. - . -. - . -. - . -. - . -. - . -


Data mining is the automatic extraction of unknown and potentially interesting information from large quantities of data. One of the major fields of data mining consists in discovering patterns occurring frequently (i.e. more than a given threshold) in data. 
%The data can either be unstructured data (sets of items like supermarket transactions for example) or data having a sequence, tree or graph structure (graph structured molecules for example).
The group works on pattern mining in complex data such as sequences, trees or graphs, which are found in many applications in chemistry (e.g. graphs representing molecules) or in bioinformatics (e.g. gene regulation networks).

This research focused on improving frequent pattern mining algorithms, both to make them more scalable and to apply them to real data analysis contexts. Main results are:

\begin{itemize}
\item  From the scalability point of view, we acquired a strong expertise on exploiting multicore processors for pattern mining [2, 5].
The proposed algorithms are also based on the notion of closed patterns [2, 4, 5], reducing the output size (hence the computation time) without loss of information.

\item These works culminated with the proposition of ParaMiner [2], the first parallel and generic algorithm for mining closed patterns.

\item From an application point of view, works have been done on the analysis of execution traces, in collaboration with STMicroelectronics.
They have improved the way to discover periodic behaviors and their disruption in traces [4], to rewrite a trace with a few significant sequences of events [3], and to automatically discover hotspots of memory contention in a parallel code [1].
\end{itemize}

\begin{description}

\item[Contracts:] FUI SoCTrace, leader for UJF partner
  
\item[Key references:]~% (if necessary)
%\cite{di:bbg+09} 

[1] Sofiane Lagraa, Alexandre Termier, Fr{\'e}d{\'e}ric P{\'e}trot: Data mining MPSoC simulation traces to identify concurrent memory access patterns. DATE 2013: 755-760

[2] Benjamin N{\'e}grevergne, Alexandre Termier, Marie-Christine Rousset, Jean-Francois M{\'e}haut: ParaMiner: a generic pattern mining algorithm for multi-core architectures, Data Mining and Knowledge Discovery, 2013

[3] Christiane Kamdem Kengne, Leon Constantin Fopa, Alexandre Termier, Noha Ibrahim, Marie-Christine Rousset, Takashi Washio, Miguel Santana: Efficiently rewriting large multimedia application execution traces with few event sequences. KDD 2013: pp 1348-1356

4] Patricia Lopez-Cueva, Aur{\'e}lie Bertaux, Alexandre Termier, Jean-Francois M{\'e}haut, Miguel Santana: Debugging Embedded Multimedia Application Traces through Periodic Pattern Mining, EMSOFT 2012 : pp 13-22

[5] Trong Dinh Thac Do, Anne Laurent, Alexandre Termier: PGLCM: Efficient Parallel Mining of Closed Frequent Gradual Itemsets, ICDM 2010 : pp 138-147

\end{description}

%. - . -. - . -. - . -. - . -. - . -. - . -. - . -. - . -. - . -. - . -. - . -. - . -. - . -. - . -. - . -. - . -. - . -. - . -. - . -. - . -. - . -
\subsubsection*{3- Social Web Data access}
%. - . -. - . -. - . -. - . -. - . -. - . -. - . -. - . -. - . -. - . -. - . -. - . -. - . -. - . -. - . -. - . -. - . -. - . -. - . -. - . -. - . -


Research has been done on new exploration problems to find useful user groups in collaborative rating datasets [1,2,4] and useful information in online news [3,5]. Our formulation of exploration as an optimization problem where various dimensions such as similarity, diversity, and coverage are optimized, leads to reductions from well-known problems and adaptations of well-established algorithms. Large-scale user studies have been conducted to verify the effectiveness of our findings. The current research direction is to blend efficient mining with exploration and to develop an evaluation methodology for large-scale information exploration.

Social Web Data access also concerns optimization. Data is stored within data centers, which constitute distributed systems. It is therefore important to optimize communications between the machines to avoid saturating the network equipment. A first work considered the problem of data routing between users of social networks. The key idea was to identify hubs that aggregate data from several sources and reduce the number of messages exchanged [6]. A second work considers the problem of data placement in hierarchical network structures. A reactive algorithm monitors data access patterns to identify locations in which new replicas of data should be deployed to reduce routers saturation [7].

\begin{description}

\item[Contracts:]  
Datalyse: Big Data Models and Algorithms, Investissement d'Avenir 2013-2016; ALICIA: ANR 2014-2017.  AGIR 2013.
  
\item[Key references:]~% (if necessary)
%\cite{di:bbg+09} 

[1]   Behrooz Omidvar Tehrani, Sihem Amer-Yahia, Alexandre Termier, Aur{\'e}lie Bertaux, Eric Gaussier, Marie-Christine Rousset: Towards a Framework for Semantic Exploration of Frequent Patterns. IMMoA 2013: 7-14 (workshops) 

[2]  Mikalai Tsytsarau, Sihem Amer-Yahia, Themis Palpanas: Efficient sentiment correlation for large-scale demographics. SIGMOD Conference 2013: 253-264

[3]  Sofiane Abbar, Sihem Amer-Yahia, Piotr Indyk, Sepideh Mahabadi: Real-time recommendation of diverse related articles. WWW 2013: 1-12

[4]  Mahashweta Das, Saravanan Thirumuruganathan, Sihem Amer-Yahia, Gautam Das, Cong Yu: Who Tags What? An Analysis Framework. PVLDB (11): 1567-1578 (2012)

[5]  Demo: Sihem Amer-Yahia, Samreen Anjum, Amira Ghenai, Aysha Siddique, Sofiane Abbar, Sam Madden, Adam Marcus, Mohammed El-Haddad: MAQSA: a system for social analytics on news. SIGMOD Conference 2012: 653-656

[6] Aristides Gionis, Flavio P. Junqueira, Vincent Leroy, Marco Serafini and Ingmar Weber: Piggybacking on social networks. In Proceedings of the 39th International Conference on Very Large Databases (VLDB, pages 409-420, 2013

[7]  Xiao Bai, Arnaud J{\'e}gou, Flavio P. Junqueira and Vincent Leroy:  DynaSoRe: Efficient In-Memory Store for Social Applications. In Proceedings of the 14th International Middleware Conference (Middleware)  pages 425-444, 2013

\end{description}

%. - . -. - . -. - . -. - . -. - . -. - . -. - . -. - . -. - . -. - . -. - . -. - . -. - . -. - . -. - . -. - . -. - . -. - . -. - . -. - . -. - . -
\subsubsection{Composing data services on the fly }
%. - . -. - . -. - . -. - . -. - . -. - . -. - . -. - . -. - . -. - . -. - . -. - . -. - . -. - . -. - . -. - . -. - . -. - . -. - . -. - . -. - . -


Composing services  exported by different organisations is a key issue when building  large scale and data-intensive systems. 
Composition must take into account the characteristics of  execution environments (e.g., memory and computing, and network capabilities) to dynamically  compose  services, and then to adapt  compositions depending on the availability and evoluation of services. 
 Compositions can be executed  on platforms providing  unlimited resources through a "pay as U go model", aware of energy consumption or services reputation, provenance, availability, and reliability.  These features  guide the way compositions are specified and executed for fulfilling given user requirements and preferences. These properties are modelled as  non functional aspects and QoS (quality of service) criteria  that can provide guarantees to the execution of compositions and to the way results are delivered.

Our research in this topic contributes to the construction of service based data management systems as service compositions. Once data management is delivered as a service, it can have associated non-functional properties.
We proposed methodologies, algorithms, languages and tools for designing and executing  service compositions with non-functional properties expressed as policies. 

We  applied our approach for the efficient evaluation  of queries as coordinations of services, including data  and computing services.  
This lead to the following results: 
\begin{itemize}
\item  an Hybrid query model for expressing queries as data service coordinations based on workflows. The approach uses the abstract state machines (ASM) formalism for defining the model.

\item  a query language HSQL (Hybrid Services Query Languages) associated to the hybrid query model and the language MQLiST (Mashup Query Spatio Temporal Language) for integrating hybrid query results in a mashup. Both are extensions of SQL. 

\item  an algorithm BP GYO for generating the query workflow that implements a query expressed in HSQL.

% An algorithm for computing computing the query workflows search space that implement an hybrid query and that respect an associated SLA expressed as an aggregation of measures (economic, temporal and energetic costs).

\item an hybrid query evaluation engine HYPATIA.

\item an Active Policy model and language for specifying the QoS properties to be associated to service compositions modelled as workflows; and inforcement actions when they are not verified.

\end{itemize}



\begin{description}

\item[Contracts:] 
OPTIMACS (2009-2012)	Agence Nationale de la Recherche, France, Programme ARPEGE LIG, LAMIH, LIRIS Service composition based framework for optimizing queries. 

CLEVER (2011-2013) STICAMSUD program  U. de la Rep{\'u}blica, Uruguay, UFRN Brazil, LIG, LAFMIA, LIFO, France
Environment virtual observatory on cloud. 

SWANS (2014-2016)	CNRS STiC-AMSUD Program U. de la Rep{\'u}blica, Uruguay, UFRN Brazil, LIG, LAFMIA, LIFO, France

(2013-2014)	COST ICT Program : Semantic keyword-based search on structured data sources. 

SOGrid (2013-2017)	ADEME Le r{\'e}seau {\'e}lectrique de demain. 

%SITAM (2012)	Mexican Council of foreign affairs, Mexique Laboratorio Nacional de Informatica Aplicada, LAFMIA, UDLAP Summit Information Technologies in Mexico. Participant

AIWS (2012 - 2014)	PEPS  CNRS program (LIRIS-LIG) Discovering conversations among services by analyzing event logs.

CAISES (2012 - 2015)	European Union FP7, IRESES program (UK, France, Ukrania, China) Observation and industrial management on the cloud.  

%S2EUNET (2010 - 2014)	European Union FP7, IRESES program Norway, Spain, USA, Mexico, China Access to continuous data on heterogeneous networks. Participant
  
\item[Key references:]~% (if necessary)
%\cite{di:bbg+09} 


[1] V. Cuevas-Vicenttin, G.  Vargas-Solar, C. Collet, Evaluating Hybrid Queries through Service Coordination in HYPATIA, In Proceedings of the 15th International Conference on Extending Database Technology (EDBT), Berlin, Germany, 2012

[2] T. Delot, S. Ilarri, M. Thilliez, G.  Vargas-Solar, S. Lecomte, Multi-scale query processing in vehicular networks, In Journal of Ambient Intelligence and Humanized Computing, Springer Verlag, ISSN 1868?5137, 2(3), 2011, pp. 213?226

[ 3]	Genoveva Vargas-Solar, Catarina Ferreira da Silva, Parisa Ghodous, Jos{\'e}-Luis Zechinelli-Martini, Moving energy consumption control into the cloud by coordinating services, International Journal of Computing Applications, Special Issue. December 2013.

% pour 2015- ???? 
% [ 3 ]	Juan Castrejon, Genoveva Vargas-Solar, Christine Collet, and Rafael Lozano, ExSchema: Discovering and Maintaining Schemas from Polyglot Persistence Applications, In Proceedings of the International Conference on Software Maintenance, Demo Paper, IEEE, 2013

[4] Carlos-Manuel Lopez-Enriquez, Genoveva Vargas-Solar, Jos{\'e}-Luis Zechinelli-Martini, Christine Collet, Hybrid query generation,  LANMR,  117128,2012.

[5] Valeria de Castro, Martin A. Musicante, Umberto Souza da Costa, Pl{\'a}cido A. de Souza Neto, and Genoveva Vargas-Solar, Supporting Non-Functional Requirements in Services Software Development Process: An MDD Approach, In Proceedings of the 40th International Conference on Current Trends in Theory and Practice of Computer Science,  LNCS Springer Verlag, High Tatras, Slovakia, January, 2014. 

[6] Javier A. Espinosa-Oviedo, Genoveva Vargas-Solar, Jos{\'e}-Luis Zechinelli-Martini, Christine Collet. Policy driven services coordination for building social networks based applications. In Proc. of the 8th Int. Conference on Services Computing (SCC'11), Work-in-Progress Track, Washington, DC, USA, July 2011.

\end{description}

%. - . -. - . -. - . -. - . -. - . -. - . -. - . -. - . -. - . -. - . -. - . -. - . -. - . -. - . -. - . -. - . -. - . -. - . -. - . -. - . -. - . -
\subsubsection{Reasoning on data semantics}
%. - . -. - . -. - . -. - . -. - . -. - . -. - . -. - . -. - . -. - . -. - . -. - . -. - . -. - . -. - . -. - . -. - . -. - . -. - . -. - . -. - . -

This research is focused on combining reasoning and data management for efficiently querying and linking Web data through ontologies. Ontologies are very useful in many applications to express domain-specific knowledge over data that may be incomplete, uncertain or even inconsistent because coming from autonomous data sources distributed over the Web. 

The proposed approach relies on recent complexity  results showing that the expressive power of ontologies must be limited for making tractable reasoning on data enriched with ontologies. In particular,  (several fragments of) the DL-Lite description logic  we have been studied  in the decentralized setting of P2P semantic networks. [1] designs a novel setting for robust module-based data management allowing to re-use a part of a  reference ontology-based data system as an independent module while guaranteeing that it evolves safely w.r.t both the reference  schema and its associated data. We are investigating how it applies to extract modules from the knowledge base on anatomy of My Corporis Fabrica.  [2,3 proposed a novel model of trust based on alignments between taxonomies for guiding the query answering process in P2P semantic networks. Finally,  [4]  provides a novel method that us systematic and mathematically well-founded for discovering mappings between taxonomies of classes. 

% We are also collaborating with Alexandre Termier and Noha Ibrahim on pattern-mining and semantic trace analysis. 

\begin{description}
\item[Contracts:] ANR projects: Continuum, Dataring, Qualinca, Pagoda 
  
\item[Key references:]~% (if necessary)
%\cite{di:bbg+09} 

[1] Robust Module-based Data Management. Francois Goasdou and Marie-Christine Rousset. IEEE Transactions on Knowledge and Data Engineering , Volume 25, Issue 3, March 2013, pages 648-661 

[2] Alignment-based trust for resource finding in semantic P2P networks. Manuel Atencia, Jerome Euzenat, Giuseppe Pirro and Marie-Christine Rousset. Proceedings of ISWC 2011 (10th International Semantic Web Conference) . 

[3] Trust in Networks of Ontologies and Alignments. Manuel Atencia, Mustafa Al Bakri and Marie-Christine Rousset.  Knowledge and Information Systems, to appear. 

[4] Discovery of Probabilistic Mappings between Taxonomies: Principles and Experiments Remi Tournaire, Jean-Marc Petit, Marie-Christine Rousset, and Alexandre Termier. Journal of Data Semantics (JoDS), Volume 15, pages 66-101. 

[5] Web Data Management, Serge Abiteboul, Ioana Manolescu, Philippe Rigaux, Marie-Christine Rousset, Pierre Senellart, book published by Cambridge University Press. 

\end{description}

% – synthèse du nombre de publications en termes de quantité et qualité
% – production et diffusion de logiciels ...

%. - . -. - . -. - . -. - . -. - . -. - . -. - . -. - . -. - . -. - . -. - . -. - . -. - . -. - . -. - . -. - . -. - . -. - . -. - . -. - . -. - . -
\subsubsection{Publications} % (fold)
%. - . -. - . -. - . -. - . -. - . -. - . -. - . -. - . -. - . -. - . -. - . -. - . -. - . -. - . -. - . -. - . -. - . -. - . -. - . -. - . -. - . -


% tableau recap issu de pistou
Inserer Ici tableau recapitulatif des publications 

\ \\

discussion de l'{\'e}volution des publications en terme de quantit{\'e} et qualit{\'e} entre le precedent quadriennal et 2009-2014. 

\ \\

% subsection scientific_and_technological_results (end)

\newpage
%==========================================================================
\subsection{Visibility and attractivity} % (fold)
\label{sub:hadas_visibility_and_attractivity}
%==========================================================================

% Critere C2 : Rayonnement et attractivite  academiques
% 1 page
% - Rayonnement : honneur, nominations,

%. - . -. - . -. - . -. - . -. - . -. - . -. - . -. - . -. - . -. - . -. - . -. - . -. - . -. - . -. - . -. - . -. - . -. - . -. - . -. - . -. - . -
\subsubsection{Rayonnement}
%. - . -. - . -. - . -. - . -. - . -. - . -. - . -. - . -. - . -. - . -. - . -. - . -. - . -. - . -. - . -. - . -. - . -. - . -. - . -. - . -. - . -
** Membres IUF : - Marie-Christine Rousset \\
** Chevalier de l'ordre national du m{\'e}rite: Marie-Christine Rousset (2011), Christine Collet (2010)

%\subsection*{Prizes and Awards}

%. - . -. - . -. - . -. - . -. - . -. - . -. - . -. - . -. - . -. - . -. - . -. - . -. - . -. - . -. - . -. - . -. - . -. - . -. - . -. - . -. - . -
\subsubsection*{Best Paper Awards}
%. - . -. - . -. - . -. - . -. - . -. - . -. - . -. - . -. - . -. - . -. - . -. - . -. - . -. - . -. - . -. - . -. - . -. - . -. - . -. - . -. - . -
\begin{itemize}

\item Three prices at SSSW 12 (Summer School on Ontology Engineering and the Semantic Web) 

\item Best Paper Award track Embedded Software, DATE 2014.

% \item  


\end{itemize}

% - Collaborations internationales,
%With grants and/or Joint publications\\
%\textit{Only collaborations that have not been already listed in section 5 should appear here}\\
%. - . -. - . -. - . -. - . -. - . -. - . -. - . -. - . -. - . -. - . -. - . -. - . -. - . -. - . -. - . -. - . -. - . -. - . -. - . -. - . -. - . -
\subsubsection{Principal International collaborations}
%. - . -. - . -. - . -. - . -. - . -. - . -. - . -. - . -. - . -. - . -. - . -. - . -. - . -. - . -. - . -. - . -. - . -. - . -. - . -. - . -. - . -

\begin{description}

\item[China:] Genoveva + Christophe corriger/completer SVP

\ \\

\item[Japon:] 

\end{description}

\begin{description}

\item[Osaka University:] 
A. Termier collaborates with Pr. Takashi Washio of I.S.I.R., Osaka University on graph mining algorithms. 
\item[NII:] 
A. Termier also collaborates informally way with Takeaki Uno from NII, Tokyo on the parallelisation of Pr. Uno's "LCM" algorithm.
This collaboration has been formalized with the PhD student B.  N{\'e}grevergne (co-supervised by M.-C. Rousset and A. Termier). 

\end{description}

Alex, corriger/completer STP

\ \\

\begin{description}
\item[Mexico:] the group has a long tradition (20 years) in developing cooperation actions between the Mexican and French governments in TICS. 
The cooperation of HADAS with Mexico includes the most important private and public institutions of that country: three major public research centres CINVESTAV, CICESE, INAOE, private centres like LANIA; and important universities like UDLAP, UATx, ITESM. 
The main research topics in the cooperation are services based infrastructures for managing distributed data with non-functional properties, services based query processing and flexible data storage services.  Collaboration on these topics has been formalized through projects (see below) and PhD students: A. Portila-Flores \footnote{Double diploma funded by the PROMEP Mexican program and the Foundation Jenkins.}, V. Cuevas \footnote{Funded by the project ANR - ARPEGE OPTIMACS.}, J. Espinosa-Oviedo \footnote{Funded through a CONACyT fellowship in the context of the ECOS-ANUIES project ORCHESTRA.}, Carlos-Manuel Lopez-Enriquez \footnote{Funded by the project ANR - ARPEGE OPTIMACS, the CONACyT and the Jenkins Foundation} and Juan Carlos Castrejon \footnote{Funded by an excellence fellowship of the doctoral school MSTII.}.  Ch. Collet, and G. Vargas-Solar co-advised these students and with Ch. Bobineau they also advised master students of Mexican institutions. Some of them continued their education as PhD. students in France: Lourdes A. Martinez Medina \footnote{Funded by the ANR project UBIQUEST.}. 


Since 2008, G. Vargas-Solar is deputy director of the French Mexican Laboratory in Informatics and Automatic Control (LAFMIA, UMI 3175 http://lafmia.imag.fr) an international unit of the CNRS. 
The cooperation of HADAS with Mexico and the LAFMIA has lead to scientific results and to the education of graduate students through co-advising contracts and the organisation of thematic schools. 



%-- To be revised (Geno) 
%\ \\


\item[Vietnam:] We have developed since 2000 a strong collaboration with the Hanoi University of Science and Technology, including the International Research Institute MICA (UMI CNRS), and Polytechnic Institutes of Danang and Ho-Chi-Minh City (4 doctors formed). This cooperation concerns adaptable distributed query optimization over stored data and data streams. Collaboration actions concern visits of senior scientists (2-3 times per year) and co-supervision of students.
%we have developed relations with the MICA laboratory in Hanoi (3 doctors formed) and participated to the Asian projects. 
\ \\
\item[Brazil:] 

The collaboration with Brazil includes mainly
the Universite Federal Rio Grande do Norte, department DIMAP,  equipe FORALL since 2007. This collaboration concerns service based data management on the cloud: semantic integration of data services through mashups, policy based service coordination, data processing on the cloud using map-reduce models, query langagues based on service compositions.  We priviledged applications on meteorology data and energy distribution. These commun research activities were funded by different organizations: 
Microsoft-LACCIR (e-CLOUDSS), the CNRS STICAMSUD program (CLEVER, SWANS) and by co-advising graduate students. In addition, postdoctoral and PhD and senior scientists internships in HADAS were funded by the CAPES and the CNRS.   

\item[Uruguay:] 

The collaboration with Uruguay concerns the University of La Republica since 2007. This collaboration concerns the specification of mashup languages including policies for associating quality of service attributes for retrieving and integrating data. The collaboration is done through commun projects (e-CLOUDSS, CLEVER, SWANS), co-advising of graduate students and invitations for lecturing seminars on topics related to the commun projects.

\item[Spain:] 

The collaboration with Spain conerns the region of Madrid and includes Universidad Rey Juan Carlos and Universidad Politecnica de Madrid. The collaboration concerns the specification of service oriented methodologies including non-functional properties. Collaboration actions concern visits of senior scientists, participation in PhD. viva, development of plug-ins for building a framework implementing the methodology $\pi$-SODM that we have proposed in the PhD of Placido Souza Neto. 

%other countries ??? 

%-- To be completed (all)
%\ \\

\end{description}


% - Implications dans des comite s, expertises, recrutements ...]
% - Editorial board, program commitees evaluation commiRees, ...
% subsection visibility_and_attractivity (end)

%. - . -. - . -. - . -. - . -. - . -. - . -. - . -. - . -. - . -. - . -. - . -. - . -. - . -. - . -. - . -. - . -. - . -. - . -. - . -. - . -. - . -
\subsubsection*{Contribution to the Scientific Community}
%. - . -. - . -. - . -. - . -. - . -. - . -. - . -. - . -. - . -. - . -. - . -. - . -. - . -. - . -. - . -. - . -. - . -. - . -. - . -. - . -. - . -


\begin{itemize}
\item {\it President of the conference BDA 2014}: Ch. Collet

\item {\it President of the EDBT 2013 school}: Ch. Collet

\item {\it Membre du conseil scientifique de la chaire d'excellence Smart Grids entre Grenoble INP et ERDF (2012- )}: Ch. Collet

\item {\it Membre du comit{\'e} de pilotage ANR, Mod les num{\'e}riques (2010-2013)}: Ch. Collet

\item {\it VP adjointe recherche groupe Grenoble INP}: Ch. Collet(April 2007-2012)

\item {\it Membre nomm{\'e}  du conseil scientifique INS2i - CNRS}: Ch. Collet (2010-)

\item {\it Charg{\'e}e de Mission aupr{\`e}s du LIG pour la prospective scientifique}: M.-C. Rousset 

\item {\it Membre du jury du prix de th{\`e}se Gilles Kahn (2010-2012)
(prix d{\'e}cern{\'e} par Specif et patronn{\'e} par l'Acad{\'e}mie des Sciences).}: Ch. Collet 

%\item {\it Elected president of the Mexican Society of Computer Science 2007-2009}:  G. Vargas-Solar

%\item {\it Deputy director of the UMI French Mexican Laboratory of Computer Science (LAFMIA, UMI 3571) (2008-) }:  G. Vargas-Solar

\end{itemize}

%. - . -. - . -. - . -. - . -. - . -. - . -. - . -. - . -. - . -. - . -. - . -. - . -. - . -. - . -. - . -. - . -. - . -. - . -. - . -. - . -. - . -
\subsubsection*{Management of Scientific Organisations}
%. - . -. - . -. - . -. - . -. - . -. - . -. - . -. - . -. - . -. - . -. - . -. - . -. - . -. - . -. - . -. - . -. - . -. - . -. - . -. - . -. - . -

\begin{itemize}
\setlength{\itemindent}{-0.5cm}
\setlength{\itemsep}{-0.1cm}
\item {\it President of the Extended Database Technology (EDBT) association}: Ch. Collet (2013 - ).
\item {\it Member of the Extended Database Technology(EDBT) association}: Ch. Collet (2004 - 2013) in charge of  school organization program.
\item {\it Member of the IJCAI-09 advisory committee}: M.-C. Rousset.
\item {\it Deputy director of the French Mexican Laboratory in Informatics and Automatic Control (LAFMIA, UMI 3175) (2008 - ) }:  G. Vargas-Solar.

\end{itemize}

%. - . -. - . -. - . -. - . -. - . -. - . -. - . -. - . -. - . -. - . -. - . -. - . -. - . -. - . -. - . -. - . -. - . -. - . -. - . -. - . -. - . -
 \subsubsection*{Administration of Professional Societies}
%. - . -. - . -. - . -. - . -. - . -. - . -. - . -. - . -. - . -. - . -. - . -. - . -. - . -. - . -. - . -. - . -. - . -. - . -. - . -. - . -. - . -
%. - . -. - . -. - . -. - . -. - . -. - . -. - . -. - . -. - . -. - . -. - . -. - . -. - . -. - . -. - . -. - . -. - . -. - . -. - . -. - . -. - . -
\subsubsection*{Editorial Boards}
%. - . -. - . -. - . -. - . -. - . -. - . -. - . -. - . -. - . -. - . -. - . -. - . -. - . -. - . -. - . -. - . -. - . -. - . -. - . -. - . -. - . -

\begin{itemize}
\setlength{\itemindent}{-0.5cm}
\setlength{\itemsep}{-0.1cm}
\item {\it PVLDB, publication of the Very Large Database Endowment:} Ch. Collet (2008-2010??)
\item {\it Computacion y sistemas}, G. Vargas-Solar, since 2002.
%\item {\it e-Gnosis, electronic journal}, G. Vargas-Solar, since 2004.
\item {\it ICDIM Journal special issue}, G. Vargas-Solar, since 2005.
\item {\it KER Journal special issue}, G. Vargas-Solar, since 2007.
\item {\it ActaPress Journal}, G. Vargas-Solar, since 2008.
\item {\it Interstices}: M.-C. Rousset.
\item {\it ACM Transactions on Internet Technology (TOIT)}: M.-C. Rousset,  until 2005.
\item {\it AI Communications}: M.-C. Rousset.
\item {\it Communications of the ACM }: M.-C. Rousset,  since 2009.
\item {\it ISI, Ing\'enierie des Syst\`emes d'Information}: Ch. Bobineau, 2009. 
\item {\it ACM Transactions on Computer Systems}: Ch. Bobineau, 2010.
\item {\it Technique et Science Informatiques}: Ch. Bobineau, 2014.

\end{itemize}

%. - . -. - . -. - . -. - . -. - . -. - . -. - . -. - . -. - . -. - . -. - . -. - . -. - . -. - . -. - . -. - . -. - . -. - . -. - . -. - . -. - . -
\subsubsection*{Organization of Conferences and Workshops}
%. - . -. - . -. - . -. - . -. - . -. - . -. - . -. - . -. - . -. - . -. - . -. - . -. - . -. - . -. - . -. - . -. - . -. - . -. - . -. - . -. - . -

\begin{itemize}
\setlength{\itemindent}{-0.5cm}
\setlength{\itemsep}{-0.1cm}
\item Extended Data Base Technology (EDBT) school 2009, Ch. Collet, T. Delot and G. Vargas-Solar
\item Extended Data Base Technology (EDBT) school 2013, Sihem Amer-Yahia and G. Vargas-Solar
\item French speaking conference Gestion de Donn\'ees – Principes, Technologies et Applications (BDA) 2014, Ch. Bobineau and F. Jouanot

\end{itemize}

%. - . -. - . -. - . -. - . -. - . -. - . -. - . -. - . -. - . -. - . -. - . -. - . -. - . -. - . -. - . -. - . -. - . -. - . -. - . -. - . -. - . -
\subsubsection*{Program committee members}
%. - . -. - . -. - . -. - . -. - . -. - . -. - . -. - . -. - . -. - . -. - . -. - . -. - . -. - . -. - . -. - . -. - . -. - . -. - . -. - . -. - . -

\begin{itemize}
\setlength{\itemindent}{-0.5cm}
\setlength{\itemsep}{-0.1cm}
% \item {\it Conference or journal name}, participant name, year

\item {\it International Conference on Distributed Computing Systems (ICDCS)}, Ch. Collet (2007, 2009).
\item {\it Extended Data Base Technologies (EDBT)}, Ch. Collet (2009), V. Leroy (2014).
\item {\it International Conference on Information and Knowledge Management (CIKM)}, V. Leroy (2013).
\item {\it Conférence en Recherche d'Information et Applications (CORIA)}, V. Leroy (2014).
\item {\it International Conference on Data Mining (ICDM)}, A. Termier (2009).
\item {\it SIAM International Conference on Data Mining (SDM)}, A. Termier (2009).
\item {\it International workshop on ambient data integration (ADI)}, F. Jouanot (2009).
\item {\it Bases de Donn{\'e}es Avanc{\'e}es (BDA)}, Ch. Collet (2010, 2012, 2013), Ch. Bobineau (2009, 2011), G. Vargas-Solar(2011).
\item {\it Gestion des Donn{\'e}es dans les Syst{\`e}mes d'Information Pervasifs (GEDSIP)}, Ch. Bobineau (2009, 2010), G. Vargas-Solar (2009).
\item {\it International Workshop on Data and Services Management in Mobile Environments (D2SME)}, Ch. Bobineau (2009).
\item {\it International ACM Conference on Management of Emergent Digital EcoSystems (MEDES)}, G. Vargas-Solar (2009).
\item {\it International Conference on the Applications of Digital Information and Web Technologies (ICADIWT)}, G. Vargas-Solar (2009).
\item {\it International Conference on Parallel and Distributed Systems (ICPADS) track on ``Web Services"}, G. Vargas-Solar (2009).
\item {\it IEEE IFIP Conference on e-Business, e-Services, e-Society}: Ch. Bobineau (2009).
\item {\it Database Engineering and Applications Symposium (IDEAS)}: Ch. Bobineau (2011, 2012, 2014).

\end{itemize}

\ \\
Je n'ai pas r{\'e}-actualise - A COMPLETER par chacun 
\ \\

%%. - . -. - . -. - . -. - . -. - . -. - . -. - . -. - . -. - . -. - . -. - . -. - . -. - . -. - . -. - . -. - . -. - . -. - . -. - . -. - . -. - . -
%\subsubsection*{Evaluation committee members}
%%. - . -. - . -. - . -. - . -. - . -. - . -. - . -. - . -. - . -. - . -. - . -. - . -. - . -. - . -. - . -. - . -. - . -. - . -. - . -. - . -. - . -
%
%%\subsubsection*{International expertise}
%
%\begin{itemize}
%
%\item {\it Member of the evaluation panel for the ERC starting grants, Panel 5 (Information and Communication)}, M.-C. Rousset (2007-)
%
%\item 
%{\it Member of the experts committee for projects evaluation of the LACCIR Microsoft virtual lab on TICS for Latin America:} G. Vargas-Solar (2007-)
%
%\end{itemize}
%
%. - . -. - . -. - . -. - . -. - . -. - . -. - . -. - . -. - . -. - . -. - . -. - . -. - . -. - . -. - . -. - . -. - . -. - . -. - . -. - . -. - . -
\subsubsection*{National expertise}
%. - . -. - . -. - . -. - . -. - . -. - . -. - . -. - . -. - . -. - . -. - . -. - . -. - . -. - . -. - . -. - . -. - . -. - . -. - . -. - . -. - . -

\begin{itemize}
\setlength{\itemindent}{-0.5cm}
\setlength{\itemsep}{-0.1cm}


\item {\it Vice-presidence  comit{\'e} d'{\'e}valuation scientifique  Big Data, decision, simulation, HPC}   Ch. Collet (2014-)

\item {\it ANR, Comite  de pilotage Modeles num{\'e}riques}, Ch. Collet, 2010-2013

\item  {\it Member of the Specialists Council of the Delegation of Science and Technologies in Puebla, Mexico:}, G. Vargas-Solar(2007-) 

\item  {\it Member of the executive board of the National Network of Information and Communications Technologies, Mexico:}, G. Vargas-Solar (2010-) 

\end{itemize}

%\ \\

- A COMPLETER par chacun : ajouter les comit{\'e}s d'{\'e}valuation de labo, comit{\'e}s de recrutement


\ \\



\subsection{Social, economical, and cultural impact} % (fold)
\label{sub:hadas_social_economical_and_cultural_impact}
% C3 : Impact social, e conomique et culturel
% 1 page
% - Main contracts and grants,

Results of our research have direct impact on applications dealing with huge amounts of data and resources in pervasive environments. 
They include traditional enterprise applications such as mining logs, web applications but also "e-science" applications (in astronomy, biology, earth science, etc.). 
Environments we consider are wireless sensor networks (e.g. natural environment surveillance, industrial process monitoring), peer-to-peer data sharing, application deployment and maintenance for smart grids, transports, networks ??? 

Application domains for data mining in structured data include: chemistry (e.g. molecule graphs), bioinformatics (e.g. gene or protein interaction networks), offline or online event logs mining.

For Social data  ???  

A REVOIR 
% - Main contracts and grants,

 %. - . -. - . -. - . -. - . -. - . -. - . -. - . -. - . -. - . -. - . -. - . -. - . -. - . -. - . -. - . -. - . -. - . -. - . -. - . -. - . -. - . -
\subsubsection{Main Contracts and grants}
%. - . -. - . -. - . -. - . -. - . -. - . -. - . -. - . -. - . -. - . -. - . -. - . -. - . -. - . -. - . -. - . -. - . -. - . -. - . -. - . -. - . -


%\subsubsection{External contracts and grants (Industry, European, National)}

\begin{description}

\item[Webcontent] (RNTL, The semantic web framework-2006-2009), 10 partners(CEA LIST, EADS DCS, Thales Research  $\&$ Technology, France Telecom R  $\&$ D, ADRIA D{\'e}veloppement, Soredab SAS, Exalead, New Phenix, Xyleme, INRIA-GEMO, INRA, INRIA-Mostrare, LIP6, PRISM, INRIA-InSitu, LIG, LIMSI-CNRS, GRIMM, EXMO, PSY-CO), \\
(http://www.webcontent-project.org/), 83300 \euro. Coordinator: CEA, Scientific lead in LIG: Ch. Collet and M.-C. Rousset. 
The WebContent project is creating a software platform to accommodate the tools necessary to efficiently exploit and extend the future of the Internet: the Semantic Web. The first targeted domain is the watch, a subpart of intelligence dedicated to warn the decider on the occurrence of an event or the evolution of a situation. It joins several Open Source tools to create the core of a Service Oriented Application and it defines the interface of several services that are available through several partners, either freely or through commercial licences. These services then exchange data in a formalized manner.


\item[OPTIMACS]: (ANR, program ARPEGE 2008-2011), 3 partners (LIG, LAMIH, LIRIS),  (http://optimacs.imag.fr), 227 128 \euro. Coordinator: Grenoble INP-LIG, Scientific lead in LIG: G. Vargas-Solar. 
OPTIMACS (SERVICE COMPOSITION BASED FRAMEWORK FOR OPTIMIZING QUERIES) combines hybrid query processing and services composition, addressing services composition and query processing including adaptive hybrid query optimization according to QoS criteria. OPTIMACS is an original research project that will lead to results with an important expected impact on ``modern data and services intensive systems" deployed on networks of heterogeneous devices, the so called ecosystem or dataspaces.

\item[DATARING] (ANR, Programme R{\'e}seaux du futur et services 2008-2011), 3 partners (http://www.lina.univ-nantes.fr/projets/DataRing/). 130 549 \euro. Coordinator: INRIA Nantes, Scientific lead in LIG: M.-C. Rousset. 
The DataRing project addresses the problem of P2P data sharing for online communities, by offering a high-level network ring across distributed data source owners. Users may be in high numbers and interested in different kinds of collaboration and sharing their knowledge, ideas, experiences, etc. Data sources can be in high numbers, fairly autonomous, i.e. locally owned and controlled, and highly heterogeneous with different semantics and structures. What we need then is new, decentralized data management techniques that scale up while addressing the autonomy, dynamic behavior and heterogeneity of both users and data sources.

\item[CONTINUUM] (ANR, Programme R{\'e}seaux du futur et services 2008-2011),  7 partners (I3S, LIG, SUEZ ENVIRONNEMENT, LYONNAISE DES EAUX, GEMALTO, LUDOTIC, MOBILEGOV), (http://continuum.unice.fr), 279 652 \euro. Coordinator: University of Nice, Scientific lead in LIG: F. Jouanot and M.-C. Rousset. 
CONTINUUM (CONTINUITE DE SERVICE EN INFORMATIQUE UBIQUITAIRE ET MOBILE) addresses the problem of service continuity within the long-term vision of ambient intelligence. A core problem is to achieve software adaptation to a variety of resources in dynamic and heterogeneous environments with an appropriate balance between system autonomy and human control. Three key scientific issues will be addressed: context management and awareness, semantic heterogeneity, and human control versus system autonomy. The professions related to water management is used as a business application domain.

\item[UBIQUEST] (ANR, Programme BLANC 2009-2012), 3 partners (LIG, CITI, LIAMA), 149 099 \euro. Coordinator: Grenoble INP-LIG, Scientific lead in LIG: Ch. Bobineau and Ch. Collet.
UBIQUEST(Ubiquitous Quest: declarative approach for integrated network and data management in wireless multi-hop networks) aims at integrating network and data management in dynamic ad-hoc networks. this integration will be done by giving a distributed database view of the whole network. Each node stores network and application data in a local database. Messages between nodes are queries or answers. The objective of this integration is the rapid development and deployment of applications and network protocols.

\item[Datalyse] (Investissement d’Avenir May 2013 -- November 2016), 219 950 \euro.
The aim of this project is to develop scalable algorithms for data mining and processing in collaboration with INRIA Saclay, LIFL, LIRMM and industrial partners: Eolas and B\&D. The project defines 3 use cases, all of them with industrial impact. The first use case, network analysis applied to data centers datasets, aims to provide interactive traffic monitoring interfaces including traffic aggregation over time abnormal traffic identification. The second use case, digital marketing, applied to server and application logs, aims to provide customer-centric statistics and customer engagement analysis using sequence mining. The third use case, linked open data, aims to develop a platform that integrates open data on the city of Grenoble and makes it readily available for the development of various applications.

\item[ALICIA] (ANR, February 2014 -- January 2017), 90 400 \euro.
The target of this project is the development of methods for information access and intelligent crowdsourcing in collaboration with Université Paris Sud, LTCI, Xerox, and UPS/IMT. In the context of information access (e.g. search or recommendation), building and maintaining user preference profiles helps applications satisfy diverse preferences. For intelligent crowdsourcing (e.g. data sourcing and micro-task completion), expertise profiles help better assign task to users. In both scenarios, the key challenges are that user preferences and expertise cannot be known in advance; and can rarely be explicitly declared by uses in a reliable or stable way. Consequently, preferences and expertise need to be discovered over time via a learning approach. Our project’s goal is the study of models and algorithms that rely on adaptive learning techniques to improve the effectiveness, performance, and scalability of user-centric applications.

\item[PAGODA] (ANR 12 JS02 007 01, Programme JCJC, January 2013 -- December 2016, coordinated by Meghyn Bienvenu), funding for one year post-doctoral researcher.
The aim of this project is to develop practical algorithms for ontology-based data access in collaboration with LRI, LIRMM, and the LADAF (Laboratoire d’anatomie de Grenoble). This project is centered on two challenges:
\textit{(i)} Scalability: in contrast with relational database management systems that benefit from decades of research on querying algorithms and optimizations, ontology-based data access is a young area of study, and despite important recent advances, including the identification of interesting tractable ontology languages, much work remains to be done in designing scalable OBDA query answering algorithms.
\textit{(ii)} Handling data inconsistencies: In real-world applications involving large amounts of data or multiple data sources, it is very likely that the data will be inconsistent with the ontology, rendering standard querying algorithms useless (as everything is entailed from a contradiction). Appropriate mechanisms for dealing with inconsistent data are thus crucial to the successful use of OBDA in practice, yet have been little explored thus far.


\item[Sogrid]  (2013-2017) - ADEME Le reseau  electrique de demain
%%   revoir 
SoGrid aims to Confirm the path already opened by ERDF in the technological revolution of the Smart Grid. The goal is to develop the network intelligence by defining and creating all the components appearing in the chain of communication.


\item[CASES]  2012-2015) - European Union FP7, PEOPLE program (UK, France, Ukrania, China)
Customized Advisory Services for Energy-efficient Manufacturing Systems.   

The project aimes at teaming up transcontinental researchers in the areas of sustainable manufacturing and information technologies to enrich the knowledge base and achieve research synergies to develop smart design and manufacturing services in terms of energy efficiency. The project 
%establishes an active international community and an effective communication channel for research teams from various regions of the EU, China and Ukraine to collaborate in the research area of eco-design and energy efficient manufacturing planning. It 
 integrates the complementary expertise of the European, Chinese and Ukrainian teams to devise ICT-based smart services and standards to address the multi-faceted requirements of global eco-design and sustainable manufacturing planning.

%3) To leverage the geographical characteristics of all teams for research evaluation, improvement and dissemination.


\item[Smart Energy] (2012-2014) Groupe de travail sur les Smartgrids; Participants: Grenoble INP LIG, G-SCOP, G2ELab, Gipsa-Lab

\item[CLEVER] (2012-2013)  CLEVER CLOUD-BASED LATIN-AMERICAN ENVIRONMENTAL VIRTUAL OBSERVATORY). 

The project aims at providing the underlying services that will enable the VO to personalize and manage mashed up services. The result will be a platform where climate reports coming from different providers in LATAM will be mashed up. Resulting mashups will be exported as VO tools for eventually building other mashups.

\item[QUALINCA] (2012-2015)

QUALINCA is a ANR Contint funded research project looking at developing mechanisms allowing to quantify the quality level of a bibliographical knowledge base, to improve the afore mentioned quality level, to maintain the quality when updating the knowledge base and to exploit the knowledge bases taking into account their quality levels.
This project aims to develop mechanisms to:

- describe the quality of an existing document database;

- maintain a given level of quality by controlling updates on such databases;

- improve the quality of a database;

- exploit these databases according to their level of quality.


\item[ SocTrace] (2011-2015)  FUI-Minalogic, OSEO 2011-2015

Partners: INRIA, LIG, TIMA, STMicroelectronics, Magilem, probayes

Coordinator: STMicroelectronics

The SoC-Trace project aims to develop a set of methods and tools based on traces of execution produced by multi-core embedded applications. It will allow developers to optimize and debug these applications more efficiently. Such methods and tools should become a building block for the design of embedded software, in response to the growing needs of analysis and debugging required by the industry. The technological barriers consist of a scaling problem (millions of events stored on gigabytes) and a trace understanding problem related to applications whose complexity is increasing. The project addresses the problem of controlling the volume of tracks and of developing new analysis techniques. SocTrace is composed of academic partners with related themes, and several industry partners including STMicroelectronics.

\end{description}

%. - . -. - . -. - . -. - . -. - . -. - . -. - . -. - . -. - . -. - . -. - . -. - . -. - . -. - . -. - . -. - . -. - . -. - . -. - . -. - . -. - . -
\subsubsection{Research Networks (European, National, Regional, Local)}
%. - . -. - . -. - . -. - . -. - . -. - . -. - . -. - . -. - . -. - . -. - . -. - . -. - . -. - . -. - . -. - . -. - . -. - . -. - . -. - . -. - . -

\begin{description}

\item[E-CLOUDSS:] (BUILDING E-GOVERNMENT CLOUDS USING DISTRIBUTED SEMANTIC SERVICES, Microsoft, 2007-2011, LACCIR, http://e-cloudss.imag.fr), 5 partners (CNRS LIG-LAFMIA, Fundacion Universidad de las Am{\'e}ricas, Puebla, Mexique, Universidad de la Republica de Uruguay, Uruguay, Universidade Federal do Rio Grande do Norte), 50,000 USD. Coordinator: J.L. Zechinelli Martini, LAFMIA, Scientific lead in LIG: G. Vargas-Solar). 
The objective of E-CLOUDSS is to propose an infrastructure for mashing up reliable semantic services for building e-government clouds. Mashups represent a new wave for building Web applications. E-CLOUDSS addresses the management (definition and enforcing at execution time) of non functional properties associated to services coordination for building reliable mashups. Effective ways to perform virtual executions is one of the main subjects of study of E-CLOUDSS.

\item[WebIntelligence:] (Cluster R{\'e}gional "Informatique, Signal, Logiciels embarqu{\'e}s" - 2006-2009). The project aims at organizing research on web intelligence in Rhone-Alpes. 

\item[ORCHESTRA:] (ORCHESTRATION TRANSACTIONNELLE DE SERVICES, Program: ECOS-ANUIES 2007-2011), 3 partners (Grenoble INP, Universidad Autonoma de Tlaxcala, Fundacion UNiversidad de las Am{\'e}ricas, Puebla, Mexique). Missions for Professors (Ch. Collet in 2007 and  2008, and G. Vargas in 2009) and PhD students. The objective of ORCHESTRA is to propose an infrastructure pour building transactional, secure and evolutive service-based applications. The key elements of the project are: (i) the definition of a framework (general solution) of technical services for managing the security, transactional properties and evolution of business services ; and (ii) implementation of the framework an its validation in the development of service-based applications: production chains. 

\end{description}

%. - . -. - . -. - . -. - . -. - . -. - . -. - . -. - . -. - . -. - . -. - . -. - . -. - . -. - . -. - . -. - . -. - . -. - . -. - . -. - . -. - . -
\subsubsection{Internal Funding}
%. - . -. - . -. - . -. - . -. - . -. - . -. - . -. - . -. - . -. - . -. - . -. - . -. - . -. - . -. - . -. - . -. - . -. - . -. - . -. - . -. - . -
% \textit{(BQR, MSTIQ projects, IMAG projects, CORDI, INRIA Post Docs, ARC, any funding from INRIA, Grenoble INP, UJF \ldots)}

 \begin{description}


\item[RED-SHINE:] (RELIABLY AND SEMANTICALLY INTEGRATING WEB INFORMATION BY MASHING UP DATA SERVICES, BQR Grenoble INP, 2009). 2 partners (LIG, LAFMIA-UMI 3175)  (http://lafmia.weebly.com/), 20 000\euro - one PhD grant and 4 months for inviting professors. Coordinator: Grenoble INP-LIG, Scientific lead in LIG: G. Vargas-Solar). \\
The objective of RED-SHINE is to propose an infrastructure for mashing up services using semantics and thereby integrating information from the Web. RED-SHINE will redefine and extend OQLiST for declaratively defining reliable semantic mashups. RED-SHINE addresses the management (definition and enforcing at execution time) of non functional properties (NF-P) associated to services' coordination for building reliable mashups. The objective of our work will be to propose a language for orthogonally expressing NF-P and ensuring strategies, and to specify execution strategies for adding NF-P to mashups.\\

\item[DAMOCLES:] (MSTIQ project, 2009). 2 partners (LIG, TIMA), 15 000 \euro - one year postdoc. Coordinator: Grenoble INP-LIG, Scientific lead in LIG: A. Termier). \\
DAMOCLES (DAta Mining for On Chip Low Energy Systems) aims at developing data mining algorithms for analyzing memory accesses in System-on-Chip processors, in order to optimise data placement and thus reduce energy consumption.

\ \\
\item[smart Energy:] 

Grenoble INP Project (2012-2015) - LIG, G-SCOP, G2ELab, Gipsa-Lab

\ \\

\item[WalT:] 
Grenoble INP   University  Joseph Fourrier, Programme AGIR (2013-2015) on Wireless Testbed
	
 \end{description}


% - Aspects interdisciplinaires
% - Vulgarisation




%==========================================================================
\subsection{Team Organization and life} % (fold)
\label{sub:hadas_team_organization_and_life}%==========================================================================

% C4 : Organisation et vie de l 'entite 
% 1/2 page
% - Seminaires
% - Vie scientifique de l 'equipe
% - Prise en compte des recommandations de la pre ce dente e valuation



% subsection team_organization_and_life (end)


%==========================================================================
\subsection{Training through research, educational involvement} % (fold)
\label{sub:hadas_training_through_research_educational_involvment}
%==========================================================================

% C5 : Implication dans la formation par la recherche
% 1/2 page
% - Nombre de the ses soutenues,
% - Suivi des doctorants,
% - devenir des doctorants
% - responsabilite s master ED...


Sur la periode 2009-2014: 
\begin{description}

\item[Thesis:]  11 thesis have been defended;  - On average, the thesis is done in four years. 


\item[Suivi:] regular meetings, presentations of papers, "workshop" on talks / papers, participation to schools,

\item[Devenir:]  
%Le devenir des doctorants est varie  : X dans l'industrie (Chercheur ou ingenieur R $\&$ D) ;  2 dans le superieur. 
The profesional perspectives of  students is diverse: X are in industry (researcher or engineer R  $\&$ D),  Y have academic positions

\ \\

% subsection training_through_research_educational_involvment (end)

\end{description}

%. - . -. - . -. - . -. - . -. - . -. - . -. - . -. - . -. - . -. - . -. - . -. - . -. - . -. - . -. - . -. - . -. - . -. - . -. - . -. - . -. - . -
\subsubsection*{Educational involvement} % (fold)
%. - . -. - . -. - . -. - . -. - . -. - . -. - . -. - . -. - . -. - . -. - . -. - . -. - . -. - . -. - . -. - . -. - . -. - . -. - . -. - . -. - . -

\ \\


\ \\


%==========================================================================
\subsection{Strategy and Research Project} % (fold)
\label{sub:hadas_strategy_and_research_project}
%==========================================================================

% Crite re C6 : Strategie et projet a  cinq ans
% 1 page
% - Projection pour les 5 ans a  venir


\ \\


\ \\
% subsection strategy_and_research_project (end)


%==========================================================================
\subsection{Self assessment} % (fold)
\label{sub:hadas_self_assesment}
%==========================================================================

% 1 page
% - [SWOT] Strengths, Weaknesses, Opportunies, Threats

% subsection self_assesment (end)


\ \\


\ \\

% section equipe (end)


%%% Local Variables: 
%%% mode: latex
%%% TeX-master: "master"
%%% LaTeX-command: "pdflatex -shell-escape"
%%% End: 
